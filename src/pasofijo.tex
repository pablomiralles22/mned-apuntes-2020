\section{Preliminares}

%TODO cambiar esta label y actualizar referencias
\begin{definition}\label{eqn:pvi}
    Un \emph{problema de valor inicial} (en adelante, PVI)
    es una ecuación diferencial ordinaria dada por
    $f : \Omega \subset \R \times \R^n \to \R^n$
    acompañada de una condición inicial $(t_0, \vy_0)$ del dominio de $f$
    del que se busca una solución $y$ que cumpla
    \begin{equation*}
        \left\{
        \begin{aligned}
            \vy'(t) &= f(t, \vy(t)) \\
            \vy(t_0) &= \vy_0 \\
        \end{aligned}.
        \right.
    \end{equation*}
\end{definition}

%TODO la definición no deja cuestionarse el orden, no?
% \begin{remark}
%     Los problemas que trataremos serán de orden $1$ siempre,
%     en otro caso los transformaremos en un problema de orden $1$.
% \end{remark}

A lo largo del texto hablaremos de la solución $\vy$ de un PVI,
por lo que estaremos asumiendo que existe una solución y que es única.
Esto será así en la mayoría de problemas que queramos resolver,
a poco que la función $f$ del PVI cumpla unos requisitos mínimos.

\begin{theorem}[Picard-Lindelöf]
    Sea $f(t, \vy) : \Omega \subset \R \times \R^n \to \R^n$,
    donde $\Omega$ es abierto,
    una función continua y
    localmente Lipschitziana respecto de la segunda variable.
    Entonces, dado $(t_0, \vy_0) \in \Omega$,
    existe un intervalo cerrado $[t_0 - h, t_0 + h]$
    donde existe una única solución del PVI
    \begin{equation*}
        \left\{
        \begin{aligned}
            \vy'(t) &= f(t, \vy(t)) \\
            \vy(t_0) &= \vy_0 \\
        \end{aligned}.
        \right.
    \end{equation*}
    que cumple que $(t, \vy(t)) \in \Omega$ para todo $t \in [t_0 - h, t_0 + h]$.
\end{theorem}

\begin{proof}
    La prueba de este teorema corresponde a
    un curso previo de ecuaciones diferenciales.
\end{proof}

\begin{remark}
    La condición de que $f$ sea Lipschitziana no es necesariamente
    muy restrictiva.
    Pensemos que es suficiente con que $f$ está definida sobre un convexo y
    que tenga derivada con respecto a la segunda variable continua y acotada.
    Lo que también es cierto si ese convexo es compacto.
\end{remark}

\subsection{Notación}

%TODO habría que considerar que t_i también se aproxima por tilde{t}_i, no?
Los métodos que veremos calcularán soluciones de un PVI
como una lista de pares $\{(t_0, \vw_0), \ldots, (t_n, \vw_n)\}$,
donde los $t_i$ son reales de forma que $t_0 < t_1 < \cdots < t_n$
y cada punto $\vw_i$ es una aproximación de $\vy(t_i)$,
$\vw_i \approx \vy(t_i)$.
No obstante, una implementación utilizaría números de precisión finita,
produciendo una lista $\{t_0, \tw_0), \ldots, (t_n, \tw_n)\}$,
de forma que
\begin{equation*}
    \tw_i \approx \vw_i \approx \vy(t_i).
\end{equation*}

\section{Métodos de paso fijo}

\begin{definition}
    Un método es \emph{de paso fijo $h$} si la solución
    $\{(t_i, \vw_i)\}_{i = 0,\ldots,n}$ que genera cumple
    \begin{equation*}
        t_i = t_{i-1} + h \qq{para todo $i = 1,\ldots,n$.}
    \end{equation*}
\end{definition}

Cuando resolvamos un PVI, generaremos una cantidad finita de puntos.
A menudo, nos interesará conocer la solución en un intervalo $[a, b]$,
$t_0 = a$,
para lo cual lo normal es generar puntos hasta que $t_n > b$.
En general, asumiremos que $t_n = b$.
O mejor dicho, el lector puede entender que $a$ representa $t_0$
y $b$ es $t_n$ para la solución concreta que estemos tratando.

\subsection{El método de Euler}

\begin{method}\label{met:euler}
    El \emph{método de Euler} es un es un método de paso fijo $h$ que
    a partir de un estado $(t_i, \vw_i) \approx (t_i, \vy(t_i))$,
    aproxima $\vy(t_i + h)$ como
    \begin{equation*}
        w_{i+1} = w_i + h\cdot f(t_i, w_i).
    \end{equation*}
\end{method}

A continuación presentamos una desigualdad elemental que utilizaremos
en la estimación del error del método de Euler.

\begin{lemma}\label{lma:tangent-exponential}
    Se cumplen las siguientes desigualdades:
    \begin{enumerate}
        \item $1 + x \le e^x$ para todo $x \in \R$.
        \item $0 \le (1+x)^m \le e^{mx}$ para todo $m > 0$ y $x \ge -1$.
    \end{enumerate} 
\end{lemma}

\begin{proof}
    Es claro que la segunda desigualdad es consecuencia de la primera.
    La primera tiene una demostración geométrica sencilla,
    puesto que $1 + x$ es la línea tangente en $x = 0$ a $y = \nume^x$,
    y las tangentes se encuentran por debajo de una función convexa.
    Resultado que se puede probar usando un desarrollo de Taylor.
    \begin{equation*}
        \nume^x = 1 + x + \frac{x^2}{2}\nume^\xi \qq{con $\abs{\xi} < \abs{t}.$}
        \implies \nume^x \ge 1+ x.
    \end{equation*}
    Donde hemos usado que $\frac{x^2}{2}\nume^\xi \ge 0$.
\end{proof}

%TODO discutir sobre la convexidad en este resultado
\begin{theorem}[Convergencia del método de Euler]
    Sean $f : D \subset \R \times \R \to \R$ con $D$ abierto convexo
    y $\{(t_i, w_i)\}_{i = 0,\ldots, n}$
    la solución calculada por el método de Euler para cierto paso fijo $h$
    para un PVI definido por $f$.
    Sea $y : [a, b] \to \R$ una solución del mismo PVI,
    con $(t, y(t)) \in D$, $\forall t \in [a, b]$,
    y $[t_0, t_n] \subset [a, b]$.
    Si se cumple que
    \begin{enumerate}[label=(\alph*)]
        \item $f$ es Lipschitziana respecto de la segunda variable en $D$,
        con constante de Lipschitz $L$.
        \item $y''$ existe en todo $(a, b)$ y
        su módulo está acotado por una constante $C \ge 0$.
        \item $(t_i, w_i) \in D$ para todo $i = 0,\dots,n$.
    \end{enumerate}
    entonces
    \begin{equation*}
        \max_{0 \le i \le n} \abs{y(t_i)-w_i} \le
        \nume^{(b-a)L}\abs{y(t_0) - w_0} + \qty[\frac{\nume^{(b-a)L}-1}{2L}]hC.
    \end{equation*}
\end{theorem}

\begin{proof}
    \newcommand{\err}{\operatorname{error}}

    Haciendo un desarrollo de orden uno de $y$ en $t_i$ se tiene
    \begin{gather*}
        y(t_{i+1}) = y(t_i) + hy'(t_i) + \frac{1}{2}h^2y''(\xi) =
            y(t_i) + hf(t_i, y(t_i)) + \frac{1}{2}h^2y''(\xi), \\
        w_{i+1} = w_i + hf(t_i, w_i).
    \end{gather*}
    Lo que implica que la diferencia
    \begin{equation*}
        y(t_{i+1}) - w_{i+1} =
        \qty\big(y(t_i) - w_i) + h\qty\big(f(t_i, y(t_i)) - f(t_i, w_i))
            + \frac{1}{2}h^2y''(\xi).
    \end{equation*}
    De esta última expresión se deduce,
    llamando $\err_i = \abs{y(t_i) - w_i}$ para $0 \le i \le n$
    y teniendo en cuenta que $\abs{\frac{1}{2}h^2y''(\xi)} \le \frac{1}{2}h^2C$,
    \begin{multline*}
        \err_{i+1} = \abs{y(t_{i+1}) - w_{i+1}} \le
        \abs{y(t_i) - w_i} + h\overset{\le L\abs{y(t_i) - w_i}}{
            \overbrace{\abs{f(t_i, y(t_i)) - f(t_i, w_i)}}
        }  + \frac{1}{2}h^2C \le \\
        (1 + hL)\abs{y(t_i) - w_i} + \frac{1}{2}h^2C =
        (1 + hL)\err_i + \frac{1}{2}h^2C.
    \end{multline*}

    Si ahora utilizamos esta desigualdad múltiples veces,
    podemos escribir el error como
    \begin{align*}
        \err_i \le {} & (1 + hL)\err_{i-1} + \frac{1}{2}h^2C \\
        \le {} & (1 + hL)\qty[(1 + hL)\err_{i-2} + \frac{1}{2}h^2C]
            + \frac{1}{2}h^2C \le \cdots \\
        \cdots \le {} & (1 + hL)^i\err_0
            + \overset{\textnormal{serie geométrica entre $0$ e $i-1$}}{
                \overbrace{\qty[
                    1 + (1 + hL) + (1 + hL)^2 + \cdots + (1 + hL)^{i-1}
                ]}
            } \frac{1}{2}h^2C \\
        = {} & (1 + hL)^i \err_0
            + \qty[\frac{(1 + hL)^i - 1}{hL}] \frac{1}{2}h^2C \\
        = {} & (1 + hL)^i \err_0
            + \qty[\frac{(1 + hL)^i - 1}{2L}] hC,
    \end{align*}
    y usando la desigualdad del \cref{lma:tangent-exponential},
    obtenemos la fórmula del enunciado
    \begin{equation*}
        \err_i \le
        \nume^{hLi}\err_0 + \qty[\frac{\nume^{hLi} - 1}{2L}] hC \le
        \nume^{(b-a)L}\abs{y(t_0) - w_0} + \qty[\frac{\nume^{(b-a)L}-1}{2L}]hC.
        \qedhere
    \end{equation*}
\end{proof}

\begin{theorem}[Convergencia del método de Euler con error de redondeo]
    Sean $f : D \subset \R \times \R \to \R$ con $D$ abierto convexo
    e $y : [a, b] \to \R$ una solución del un PVI dado por $f$,
    con $(t, y(t)) \in D$, $\forall t \in [a, b]$,
    y $[t_0, t_n] \subset [a, b]$.
    Fijado $h > 0$, consideramos la solución numérica
    \begin{equation}
    \begin{cases}
        w_0 = y_0 \\
        w_{i+1} = w_i + hf(t_i, w_i) + \delta_i & \text{para $i = 1,\ldots,n$.}
    \end{cases}
    \end{equation}
    con $\delta_i < \delta$ para todo $i = 0,\ldots,n$.
    Si se cumple que
    \begin{enumerate}[label=(\alph*)]
        \item $f$ es Lipschitziana respecto de la segunda variable en $D$,
        con constante de Lipschitz $L$.
        \item $y''$ existe en todo $(a, b)$ y
        su módulo está acotado por una constante $C \ge 0$.
        \item $(t_i, w_i) \in D$ para todo $i = 0,\dots,n$.
    \end{enumerate}
    entonces
    \begin{equation*}
        \max_{0 \le i \le n} \abs{y(t_i)-w_i} \le
        \nume^{(b-a)L}\abs{y(t_0) - w_0} + \qty[\frac{\nume^{(b-a)L}-1}{L}]\qty(
            \frac{1}{2}hC + \frac{\delta}{h}
        ).
    \end{equation*}
\end{theorem}

Una idea intuitiva es que
conforme reducimos el paso con el que resolvemos un PVI,
el error disminuye cada vez en menor medida,
de forma que podemos comparar dos soluciones obtenidas con pasos similares
para intentar estimar el error de una de ellas.
El siguiente teorema trata esta idea e incluso
nos permite obtener un método mejor a partir del método de Euler.

%TODO este teorema de dónde sale?
\begin{theorem}
    Dado un PVI definido por $f$
    y la solución $\{(t_i, w_i)\}_{i = 0,\ldots, n}$
    obtenida por el método de Euler,
    si la solución real $y : [t_0, t_n] \to \R$ es de clase $C^3$
    y $\pdv{f}{y},\pdv[2]{f}{y}$ son continuas,
    entonces
    \begin{equation}
        y(t_i) - w_i = hd(t_i) + \vo(h^2) \qq{para todo $t \in [t_0,b]$,}
    \end{equation}
    dónde $d$ es solución del PVI
    \begin{equation}
    \begin{cases}
        d'(t) \, = g(t)d(t) + \frac{1}{2}y''(t),
            & g(t) = \pdv{f(t,y)}{y}\qty(t, y(t)) \\
        d(t_0) = 0
    \end{cases}
    \end{equation}
\end{theorem}

\begin{proof}
    %TODO una buena referencia a la bibliografía o algo.
    Puede encontrarse en el libro de Atkinson, pero no entra.
\end{proof}

\begin{remark}
    Cuando $d(t)$ se puede obtener explícitamente,
    el término $hd(t_i)$ de la fórmula normalmente proporciona
    una estimación bastante buena del error real cometido, $\abs{y(t_i) - w_i}$,
    y la precisión de la estimación mejora si se disminuye el tamaño del paso.
\end{remark}

%TODO el remark parece decir que es práctico... te refieres a sencillo?
No suele ser muy práctico encontrar $d$,
pero el resultado nos ayuda a demostrar el siguiente teorema.

\begin{theorem}
    \newcommand{\hh}{\sfrac{h}{2}}

    En condiciones similares a las del teorema anterior,
    dadas $(t_i^h, w_i^h)$ y $(t_i^{\hh}, w_i^{\hh})$,
    dos soluciones obtenidas con el método de Euler
    con pasos $h$ y $\hh$ respectivamente,
    se cumple que
    \begin{enumerate}[label=(\alph*)]
        \item $y(t_i^h) - w_{2i}^{\hh} = (w_{2i}^{\hh} - w_i^h) + \vo(h^2)$.
        \item $w_i := 2\cdot w_{2i}^{\hh} - w_i^h$
        es una aproximación de $y$ de orden $\vo(h^2)$.
    \end{enumerate}
\end{theorem}

\begin{proof}
    \newcommand{\hh}{\sfrac{h}{2}}

    Aplicando el teorema anterior obtenemos
    \begin{align*}
        y(t_i^h) - w_i^h = hd(t_i^h) + \vo(h^2),
            & \qq{para todo $i = 1,\ldots,n$,} \\
        y(t_{2i}^{\hh}) - w_{2i}^{\hh} = \tfrac{h}{2}d(t_{2i}^{\hh})
            + \vo((\tfrac{h}{2})^2),
            & \qq{para todo $i = 1,\ldots,2n$.}
    \end{align*}
    Basta observar que $t_i^h = t_{2i}^{\hh}$
    y restar la primera ecuación a la segunda multiplicada por $2$.
    Despejando se obtiene tanto (a) como (b).
\end{proof}

\subsection{Estimación general del error}

Como hemos visto, a partir de conocer el error cometido en cada paso
hemos logrado estimar globalmente
el error de la solución generada por el método de Euler.
Antes de continuar nuestro estudio de más métodos de paso fijo,
vamos a definir el concepto de error local de truncamiento,
que iremos calculando para cada método.
A continuación incluímos la definición para un método general.

% \begin{definition}
%     Sean $f : \Omega \subset \R \times \R^n$ la función de un PVI
%     $p_i = (t_i, \vw_i)$ un estado del dominio de $f$,
%     $p_{i+1} = (t_{i+1}, \vw_{i+1})$
%     el siguiente estado generado por un método numérico e
%     $\vy_i$ una solución del PVI dado por $f$ y $p_i$.

%     Se define \emph{el error local de truncamiento de la estimación} como
%     \begin{equation*}
%         Z_{i+1} = \frac{\vy_i(t_{i+1}) - \vw_{i+1}}{t_{i+1} - t_i}.
%     \end{equation*}
% \end{definition}

La mayoría de métodos numéricos permiten obtener aproximaciones más precisas
a costa de realizar más cálculos.
Por ejemplo, el método de Euler nos permite
obtener mejores soluciones reduciendo el tamaño del paso,
a costa de tener que evaluar más veces la función $f$ numéricamente,
que es lo que más tiempo de computación necesita habitualmente.
Por tanto, el objetivo es
comparar los errores de truncamiento de diferentes métodos
conforme disminuye el tamaño del paso.

\begin{definition}
    Un \emph{método en diferencias} a un paso es un método que de paso fijo $h$
    que a partir de un estado $(t_i, \vw_i) \approx (t_i, \vy(t_i))$,
    aproxima $\vy(t_i + h)$ como
    \begin{equation*}
        w_{i+1} = w_i + h\cdot \phi(h, t_i, w_i).
    \end{equation*}
    para un tamaño de paso $h$.
\end{definition}

\begin{definition}
    Sea $\{t_i, \vw_i\}_{i = 1,\ldots,n}$ una aproximación obtenida por
    un método en diferencias de paso $h$ para un PVI
    y $\vy$ la solución de ese PVI.

    Se define \emph{el error local de truncamiento del método} como
    \begin{equation*}
        Z_{i+1}(h) = \frac{\vy(t_{i+1}) - \vy(t_i)}{h} - \phi(h, t_i, \vy(t_i)).
    \end{equation*}
\end{definition}

%TODO esta definición sería cogiendo la solución en el punto actual
% y la anterior es fijando una solución global asumiendo que el método
% no comete fallos
% \begin{definition}
%     Sean $f : \Omega \subset \R \times \R^n$ la función de un PVI
%     $p_i = (t_i, \vw_i)$ un estado del dominio de $f$,
%     $p_{i+1} = (t_i + h, \vw_{i+1})$
%     el siguiente estado generado por un método numérico que depende de $h$ e
%     $\vy_i$ una solución del PVI dada por $f$ y $p_i$.

%     Se define \emph{el error local de truncamiento del método} como
%     \begin{equation*}
%         Z_{i+1}(h) = \frac{\vy_i(t_i + h) - \vw_{i+1}}{h}.
%     \end{equation*}
%     En el caso de un método en diferencias,
%     \begin{equation*}
%         Z_{i+1}(h) = \frac{\vy_i(t_i + h) - \vw_i}{h} - \phi(h, t_i, w_i).
%     \end{equation*}
% \end{definition}

%TODO revisar
\begin{example}
    En el caso concreto del método de Euler se tiene
    \begin{equation*}
        Z_{i+1}(h) =
        \frac{\vy(t_{i+1}) - \vy(t_i)}{h} - h\cdot f(t_i,\vy(t_i)) =
        \frac{h}{2} \vy''(\xi_i),
            \qq{$\xi_i \in [t_i, t_{i+1}]$ para todo $i=0,\dots, n-1$}.
    \end{equation*}
    $\vy$ si $\abs{y''(t)} \le C$ para todo $t \in [a, b]$ entonces
    \begin{equation*}
        \abs{Z_{i+1}(h)} \le \frac{C}{2}h = \vo(h).
    \end{equation*}

    Lo que nos interesa es que este error
    decrezca lo más rápido posible con $h$,
    es decir, que sea $\vo(h^p)$ con $p$ lo más grande posible.
    En esta búsqueda llegamos al siguiente método, el de Taylor.
\end{example}

\subsection{Notación para la diferenciación en varias variables}

A continuación en el método utilizamos la notación
$\pdv[2]{\vf(t_i, \vw_i)}{y}$
para representar la matriz hessiana que saldría al considerar $\vf$
como una función de $y$ exclusivamente.

Los lectores que prefieran razonar el método
cuando $y$ es un escalar en vez de un vector,
pueden sustituir las evaluaciones de la diferencial y la hessiana
por multiplicaciones.
En el caso particular de la hessiana, saldría el término al cuadrado.

\subsection{El método de Taylor}
 
De aquí en adelante asumiremos el PVI general del principio,
y cualquier método que considere intentará aproximar una solución suya.

\begin{method}\label{met:taylor}
    \newcommand{\D}{\vb{D}}

    El \emph{método de Taylor de orden $p$} es un es un método de paso fijo $h$
    que a partir de un estado $(t_i, \vw_i) \approx (t_i, \vy(t_i))$,
    aproxima $\vy(t_i + h)$ como
    \begin{equation*}
        \vw_{i+1} =
        \vw_i + h\cdot \D^{(0)}(t_i, \vw_i)
            + \frac{h^2}{2}\D^{(1)}(t_i, \vw_i)
            + \cdots
            + \frac{h^p}{p!}\cdot \D^{(p-1)}(t_i, \vw_i) =:
        \vw_i + h\cdot \vb{T}^{(p)}(h, t_i, \vw_i),
    \end{equation*}
    donde
    \begin{align*}
        \D^{(0)}(t_i, \vw_i) &= \vf(t_i, \vw_i), \\
        \D^{(1)}(t_i, \vw_i) &=
            \pdv{\vf(t_i, \vw_i)}{t} +
            \pdv{\vf(t_i, \vw_i)}{\vy}\qty(\vf(t_i, \vw_i)), \\
        \D^{(2)}(t_i, \vw_i) &=
            \pdv[2]{\vf(t_i, \vw_i)}{t} +
            \pdv{\vf(t_i, \vw_i)}{t}{\vy}\qty(\D^{(1)}(t_i, \vw_i)) +
            \pdv[2]{\vf(t_i, \vw_i)}{\vy}\qty(
                \vf(t_i, \vw_i), \vf(t_i, \vw_i)
            ) \\
        \D^{(3)}(t_i, \vw_i) &= \cdots
    \end{align*}
\end{method}

\begin{remark}
    La expresión de $\vb{D}^{(i)}$ en el método de Taylor surge
    de de derivar la función $\vy'(t) = \vf(t, \vy(t))$ múltiples veces.
    Mientras que por $\vb{T}^{(p)}$ denotamos al
    polinomio de Taylor de $\vy'$.
\end{remark}

Unas pocas cuentas son suficientes para darse cuenta de que
calcular estas derivadas de manera genérica es realmente costoso,
y normalmente se suelen hace las cuentas de manera \emph{adhoc}
para un problema concreto.

%TODO un ejemplillo aquí taría de puta madre

El siguiente teorema era de esperar.

\begin{theorem}
    Si $y$ es de clase $C^{(p+1)}([a, b])$,
    el ELT del método de Taylor de orden $p$ es $\vo(h^p)$.
\end{theorem}

\begin{proof}
    En estas condiciones el ELT no es más que el resto de Lagrange:
    \begin{equation*}
        Z_{i+1}(h) = \frac{h^p}{(p+1)!} \vf^{(p)}(\xi_i, y(\xi_i)),
            \qq{$\xi_i \in [t_i, t_{i+1}]$ para todo $i = 1,\ldots n$.}
    \end{equation*}
    Y si $f(t, y(t))$ está definida sobre un compacto
    entonces está acotada en $[a,b]$,
    y por lo tanto $Z_{i+1}(h) = \vo(h^p)$.
\end{proof}

%TODO esto solo para una dimensión?

A pesar de este gran resultado respecto al error,
la dificultad para aplicarlo de manera genérica y
los problemas que conlleva la derivación numérica no lo hacen muy práctico.
Buscamos entonces métodos que no necesiten estas derivadas
y que tengan órdenes similares.

\subsection{Métodos de Runge-Kutta}

\newcommand{\vbeta}{\vb*{\beta}}

Comenzamos buscando métodos de orden $2$,
para lo cual queremos aproximar
$\vb{T}^{(2)}(t, y) = f(t,y) + \frac{h}{2} f'(t,y)$
con un error de orden $\vo(h^2)$.
Una idea feliz es considerar una aproximación de la forma
$a_1\vf(t + \alpha_1, \vy + \vbeta_1)$.
Aproximando esto último por Taylor tenemos
\begin{multline*}
    a_1\vf(t + \alpha_1, y + \vbeta_1) = \\
    a_1\qty(
        \vf(t,y) +
        \pdv{\vf(t, \vy)}{t} \cdot \alpha_1 +
        \pdv{\vf(t, \vy)}{\vy} \qty(\vbeta_1) +
        \frac{1}{2}\pdv[2]{\vf(\xi, \mu)}{t} \cdot \alpha_1^2 +
        \pdv{\vf(\xi, \mu)}{t}{\vy} \qty(\vbeta_1) \cdot \alpha_1 +
        \frac{1}{2}\pdv[2]{\vf(\xi, \mu)}{\vy} \qty(\vbeta_1, \vbeta_1)
    ).
\end{multline*}

Nótese que suponemos suficiente regularidad en $f$
para que las parciales cruzadas sean iguales.
Haciendo $a_1 = 1$, $\alpha_1 = \frac{h}{2}$, $\vbeta_1 = \frac{h}{2} \vf(t,y)$
la igualdad anterior se traduce en
\begin{multline*}
    f(t + \frac{h}{2}, y + \beta_1) = \\
    \vf(t,y) +
        \pdv{\vf(t, \vy)}{t} \cdot \frac{h}{2} +
        \pdv{\vf(t, \vy)}{\vy} \qty(\vf(t, y)) +
        \frac{h^2}{8}\pdv[2]{\vf(\xi, \mu)}{t} +
        \frac{h^2}{4}\pdv{\vf(\xi, \mu)}{t}{\vy} \qty(\vf(t, y)) +
        \frac{h^2}{8}\pdv[2]{\vf(\xi, \mu)}{\vy} \qty(\vf(t, y), \vf(t, y)) = \\
    \vb{T}^{(2)}(t, y) + \frac{h^2}{8}\qty(
        \pdv[2]{\vf(\xi, \mu)}{t} +
        2\pdv{\vf(\xi, \mu)}{t}{\vy} \qty(\vf(t, y)) +
        \pdv[2]{\vf(\xi, \mu)}{\vy} \qty(\vf(t, y), \vf(t, y))
    ).
\end{multline*}

De nuevo bajo buenas condiciones de $f$
lo que hay entre paréntesis está acotado,
y al estar multiplicado por $h^2$ es de orden $\vo(h^2)$.
Hemos obtenido nuestro primer método de Runge-Kutta.

\begin{method}\label{met:euler}
    El \emph{método del punto medio} es un es un método de paso fijo $h$ que
    a partir de un estado $(t_i, \vw_i) \approx (t_i, \vy(t_i))$,
    aproxima $\vy(t_i + h)$ como
    \begin{equation*}
        \vw_{i+1} = \vw_i + h\cdot \vf\qty(
            t + \tfrac{h}{2}, \vy + \tfrac{h}{2}\vf(t, \vy)
        ).
    \end{equation*}
\end{method}

\begin{remark}
    Para este método solo necesitamos la función $\vf$,
    pero necesitamos dos evaluaciones en cada paso.
\end{remark}

\begin{remark}
    Al ser $\vf\qty(
        t + \tfrac{h}{2}, \vy + \vf(t + \tfrac{h}{2}, \vy + \tfrac{h}{2}\vy)
    ) = \vb{T}^{(2)}(t, \vy) + \vo(h^2)$,
    el ELT será
    \begin{multline*}
        Z_{i+1}(h) =
        \frac{\vy(t_{i+1}) - \vy(t_i)}{h} - \qty(
            T^{(2)}(t_i, \vy(t_i)) + \vo(h^2)
        ) =
        \qty(
            \frac{\vy(t_{i+1}) - \vy(t_i)}{h} - T^{(2)}(t_i, \vy(t_i))
        ) - \vo(h^2) = \\
        \vo(h^2) - \vo(h^2) = \vo(h^2).
    \end{multline*}
\end{remark}

Podríamos intentar una estrategia parecida para $\vb{T}^{(3)}(t, \vy)$,
pero nuestros intentos fracasan porque
(haciendo las cuentas con escalares)
si intentamos aproximarlo mediante
\begin{equation}\label{eqn:t3-aproximation}
    T^{(3)}(t,y) =
    f(t,y) + \frac{h}{2}\dv{f(t,y)}{t} + \frac{h^2}{6}\dv[2]{f(t,y)}{t} \approx
    a_1f(t,y) + a_2f(t+\alpha_2, y+\delta_2f(t,y))
\end{equation}
pero ni siquiera esto basta para igualar el término
$\frac{h^2}{6}\qty[\pdv{f(t, y)}{y}]^2f(t,y)$
que resulta al desarrollar $\frac{h^2}{6}\dv[2]{f(t, y)}{t}$.
Invitamos al lector a hacer las cuentas.
Obtenemos sin embargo más métodos de orden $2$.

\begin{method}
    El \emph{método modificado de Euler} se obtiene de
    tomar $a_1 = a_2 = \frac{1}{2}$ y $\alpha_2 = \delta_2 = h$
    en la ecuación  \eqref{eqn:t3-aproximation}.
    \begin{equation*}
        \vw_{i+1} = \vw_i + \frac{h}{2}\qty\Big[
            \vf(t_i, \vw_i) + \vf(t_i + h, \vw_i + h\vf(t_i,\vw_i))
        ].
    \end{equation*}
\end{method}

\begin{method}
    El \emph{método de Heun} se obtiene de tomar $a_1 = \frac{1}{4}$,
    $a_2 = \frac{3}{4}$ y $\alpha_2 = \delta_2 = \frac{2}{3}h$
    en la ecuación \eqref{eqn:t3-aproximation}.
    \begin{equation*}
        \vw_{i+1} = \vw_i + \frac{h}{4}\qty\Big[
            \vf(t_i, \vw_i) +
            3\vf(t_i + \tfrac{2}{3}h, \vw_i + \tfrac{2}{3}h\vf(t_i, \vw_i))
        ].
    \end{equation*}

\end{method}

El método de Runge-Kutta más utilizado es el siguiente, de orden $4$.

\begin{method}\label{met:euler}
    \newcommand{\K}{\vb{K}}

    El \emph{método de Runge-Kutta de orden $4$}
    es un es un método de paso fijo $h$ que
    a partir de un estado $(t_i, \vw_i) \approx (t_i, \vy(t_i))$,
    aproxima $\vy(t_i + h)$ como
    \begin{gather*}
        \vw_{i+1} = \vw + \frac{
            \K_1^{i+1} + 2\K_2^{i+1} + 2\K_3^{i+1} + \K_4^{i+1}
        }{6}. \\
        \begin{array}{ll}
        \K_1^{i+1} = h\vf\qty\bigg(t, \vw_i), &
        \K_2^{i+1} = h\vf\qty(t + \frac{h}{2}, \vw_i+ \frac{\K_1^{i+1}}{2}), \\
        \K_3^{i+1} = h\vf\qty(t + \frac{h}{2}, \vw_i+ \frac{\K_2^{i+1}}{2}), &
        \K_4^{i+1} = h\vf\qty\bigg(t + h, \vw_i + \K_3^{i+1}).
        \end{array}
    \end{gather*}
\end{method}

El lector puede haber empezado a intuir que cuanta más precisión queremos
más evaluaciones de $f$ necesitamos.
Una pregunta interesante es
el número mínimo de evaluaciones que necesitaríamos en cada paso
para conseguir una aproximación de cierto orden.
La \cref{tab:butcher} relaciona ambos conceptos.

\begin{table}[H]
    \centering
    \begin{tabular}{|l||l|l|l|l|}
        \hline
        Evaluaciones & $1\le p\le 4$ & $5\le p\le 7$ & $8\le p\le 9$
            & $10\le p$ \\
        \hline
        Mejor ELT & $\vo(h^p)$ & $\vo(h^{p-1})$ & $\vo(h^{p-2})$
            & $\vo(h^{p-3})$ \\
        \hline
    \end{tabular}
    \caption{Tabla de Butcher.
        Muestra, para un número $p$,
        el máximo orden de un método de Runge-Kutta con $p$ evaluaciones
    }
    \label{tab:butcher}
\end{table}

Para muchas aplicaciones,
se utiliza el número de evaluaciones de la función $f$
como una medida de la complejidad,
pues puede ser una función muy costosa.
Con nuestros avances actuales tenemos también dos formas de mejora
de la precisión de la solución numérica:
reducir el paso
y usar un método de orden mayor.
La \cref{tab:order-vs-step} muestra la comparación de estas dos formas
con los métodos de Runge-Kutta vistos hasta el momento,
fijando el número de evaluaciones a $40$
y suponiendo un intervalo con $b - a = 1$.
En general obtenemos mejores resultados aumentando el orden del método.

%TODO esto para que problema
\begin{table}[H]
    \centering
    \begin{tabular}{|l|l|l|}
        \hline
    Método           & Paso  & Precisión aproximada \\
        \hline
        \hline
    RK4              & 0.1   & $10^{-4}$ \\
        \hline
    Euler Modificado & 0.05  & $2.5\cdot 10^{-3}$ \\
        \hline
    Euler            & 0.025 & $2.5\cdot 10^{-2}$ \\
        \hline
    \end{tabular}
    \caption{Tres métodos de Runge-Kutta;
        de órdenes $1$, $2$ y $4$;
        aplicados al mismo problema ajustando el paso
        para hacer el mismo número de evaluaciones de la función $f$.
    }
    \label{tab:order-vs-step}
\end{table}

\section{Métodos de paso adaptativo}
\subsection{Control del error global a través del criterio del error local}
\begin{definition}
Dado el problema de valor inicial 
$$
\left\{
\begin{array}{lll}
y'(t) = f(t,y) & & \\
y(t_0) = y_0 & &
\end{array}
\right.
$$
 que queremos resolver, y un método numérico en diferencias
\begin{equation}
\label{eq6}
\left\{
\begin{array}{lll}
\omega_0 = \alpha & & \\
\omega_{i+1} = \omega_i + h \phi(t_i, \omega_i, h) & & 
\end{array}
\right.
\end{equation}
definimos la \textbf{solución local} $z_n$ como la solución del pvi
$$
\left\{
\begin{array}{lll}
z_n'(t) = f(t,z_n) & & \\
z_n(t_n) = \omega_n & &
\end{array}
\right.
$$
\end{definition}

Este último pvi es el que realmente resolvemos en cada paso que damos en un método:
\begin{figure}[H]
    \centering
    \begin{tikzpicture}[scale=1.5]
          \draw[->] (-0.5, 0) -- (3,0) node[right] {$t$};
          \draw[->] (0,-0.5) -- (0,2.5) node[above] {$y$};
          \draw [fill,red] (0.5,0.5) circle [radius=0.05] node[red,right] {$t_0$}; %primer punto del método
          \draw [fill, green] (1,1.3) circle [radius=0.05] node[green,right] {$t_1$}; %2º punto del método
          \draw [fill, green] (1.7,2) circle [radius=0.05] node[green,right] {$t_2$}; %3º punto del método
          
          \draw[red,-] (0.5,0.5)to[bend left] (3,1.5) ;
          \draw[green,-,dashed] (0.5,0.85)to[bend left] (3,1.8) ;
          \draw[green,-,dashed] (0.5,1.15)to[bend left] (3,2.05) ;
          \draw[green,-] (0.5,0.5) -- (1,1.3) ;
          \draw[green,-] (1,1.3) -- (1.7,2) ;
\end{tikzpicture}
    \caption{Pvi en cada paso 1D}
\end{figure}
que en este dibujo serían las curvas de color verde (el problema inicial sería la curva roja).
\begin{theorem}
Supongamos que el método de integración  (\ref{eq6}) verifica (en cada paso)
$$
\begin{array}{ll}
\|z_n(t_n+h) - \omega_{n+1}\| \leq \varepsilon\cdot h & \textnormal{(criterio de error local)}
\end{array}
$$
para cada tolerancia $\varepsilon$ dada.\\
Entonces, si $f$ es lipschitziana con constante $k$, se tiene que 
$$
\|y(t_n)-\omega_n\| \leq e^{k(t_n-a)}\|y(b)-\omega_0\| + \frac{e^{k(t_n-a)}}{k}\varepsilon.
$$
\end{theorem}
\begin{remark}
Recuerda a la cota que obtuvimos para el método de Euler (si $k$ no es muy grande, para $b-a\approx 1$ es aceptable).
\end{remark}
\begin{remark}
Es independiente del método.
\end{remark}
\begin{remark}
Observar que, si asumimos que $z_n(t_n) \approx y(t_n) \approx \omega_n $, este criterio consiste en mantener
$$
\|\tau_{i+1}(h)\| = \frac{\| y(t_{i+1}) - (y(t_i) + h\phi(t_i, \omega_i, h)) \|}{h} \approx \frac{\| z_i(t_{i+1}) - \omega_{i+1} \|}{h} \leq \varepsilon
$$
es decir, mantener el error local de truncamiento por debajo de la tolerancia.
\end{remark}

\newpage
\subsection{Control del error por extrapolación de Richardson}
\begin{theorem}
Supongamos que el método en diferencias 
\begin{equation*}
\left\{
\begin{array}{lll}
\omega_0 = \alpha & & \\
\omega_{i+1} = \omega_i + h_i \phi(t_i, \omega_i, h_i) & & i = 0,1,\dots, N-1
\end{array}
\right.
\end{equation*}
verifica (en todo paso): \astfootnote{Usando la notación $\omega_{i+1} = y_h(t_i+h)$.}
\begin{equation}
\label{eq7}
z_i(t_i+h) = y_h(t_i+h) + c\cdot z_i^{(k+1)}(t_i)h^{k+1} + \mathcal{O}(h^{k+2})
\end{equation}
donde $c$ es una constante. Entonces se tiene:
\begin{equation}
\label{eq8}
z_i(t_i+h) = y_{\frac{h}{2}}(t_i+h) + 2c\cdot z_i^{(k+1)}(t_i)\left(\frac{h}{2}\right)^{k+1} + \mathcal{O}(h^{k+2})
\end{equation}
\end{theorem}
\begin{remark}
En este caso el error local de truncamiento es de orden $k$. Por ejemplo, el método de Euler tiene $k=1$ y el de Runge-Kutta de orden 4 tiene $k=4$.
\end{remark}

Haciendo $(\ref{eq8})\cdot 2^k - (\ref{eq7})$ obtenemos una fórmula para \textbf{mejorar la solución}:
\begin{equation}
\label{eq9}
z_i(t_i+h) = \frac{2^ky_{\frac{h}{2}}(t_i+h)-y_h(t_i+h)}{2^k-1}+\mathcal{O}(h^{k+2})
\end{equation}
Y restando a eso $y_h(t_i+h)$, obtenemos una fórmula para \textbf{aproximar el error local de truncamiento}:
\begin{equation}
\label{eq10}
z_i(t_i+h) -y_h(t_i+h)= \frac{2^k}{2^k-1}\left(y_{\frac{h}{2}}(t_i+h)-y_h(t_i+h)\right)+\mathcal{O}(h^{k+1})
\end{equation}
\begin{example}
\begin{itemize}
    \item Euler ($k=1$)  
    $$
    elt \approx 2\left(y_{\frac{h}{2}}(t_i+h)-y_h(t_i+h)\right)
    $$
    \item RK4 ($k=4$).
    $$
    elt \approx \frac{16}{15}\left(y_{\frac{h}{2}}(t_i+h)-y_h(t_i+h)\right)
    $$
\end{itemize}
\end{example}

\newpage
\subsection{Estimación local del paso con Richardson}
Un método ideal de la ecuación de diferencias 
\begin{equation*}
\left\{
\begin{array}{lll}
\omega_0 = \alpha & & \\
\omega_{i+1} = \omega_i + h_i \phi(t_i, \omega_i, h_i) & & i = 0,1,\dots, N-1
\end{array}
\right.
\end{equation*}
para aproximar la solución $z_i(t)$ al problema de valor inicial
\begin{equation*}
\left\{
\begin{array}{lll}
z_i'(t) = f(t,z_i) & & \\
z_i(t_i) = \omega_i & &
\end{array}
\right.
\end{equation*}
deberá tener la propiedad de que, con una tolerancia $\varepsilon >0$, se puede mantener $$|z_i(t_i+h) - y_h(t_i+h)| = |z_i(t_i+h) - \omega_{i+1}| \leq \varepsilon\cdot h$$ para todo $i=0,1,\dots N-1$ si queremos controlar el error local, ya que este es el error introducido en la solución en el punto $t_{i+1} = t_i +h$ si asumimos que la solución $y_n$ en el punto anterior, $t_n$, es la solución exacta. \\
La manera más fácil de estimar este error es usando la extrapolación de Richardson de la siguiente manera: 
\begin{enumerate}
\item Resolvemos el pvi dos veces en el intervalo $[t_0,b]$ con tamaños de paso $2h$ y $h$ \astfootnote{P.Esquembre usa $h$ y $\frac{h}{2}$.}.
\item Estimamos el error, $\tilde{e}_r(h) = \| z_i(t_i+h_i) - \omega_{i+1} \|$, con extrapolación de Richardson.
\item Si el error estimado es menor o igual que $\varepsilon \cdot h$ ($\tilde{e}_r(h)\leq \varepsilon \cdot h$) entonces el paso es el adecuado y pasamos a aproximar $\omega_{i+1}$ con la fórmula (\ref{eq9}).
\item Si no, ¿cuál tendría que haber sido el paso $h^* = qh$ para que el error sí fuese menor o igual que $\varepsilon \cdot h$? Lo vemos:
$$
\|z_i(t_h + qh) - y_{qh}(t_i+qh)\| \approx C(qh)^{k+1} = Cq^{k+1}h^{k+1} \approx
$$
$$
\approx q^{k+1}\cdot \|z_i(t_h + h) - y_h(t_i+h)\| = q^{k+1}\cdot \tilde{e}_r(h)\leq \varepsilon qh
$$
\\Donde hemos usado que $Ch^{k+1} \approx \|z_i(t_i + h) - y_h(t_i+h)\|$. Obtenemos, entonces, que debe ser
$$
q \leq \left(\frac{\varepsilon h}{\tilde{e}_r(h)}\right)^{\frac{1}{k}}
$$
Por tanto, cuando $\tilde{e}_r(h)> \varepsilon \cdot h$ lo que haremos será tomar $q =\left(\frac{\varepsilon h}{\tilde{e}_r(h)}\right)^{\frac{1}{k}} $ y $h = q\cdot h$ para volver a dar el paso con un $h$ que esperamos nos permita mantener $\tilde{e}_r(h)\leq \varepsilon\cdot h$. Este proceso se repetirá hasta que lleguemos a un $h$ que cumpla esto último.
\end{enumerate}
\begin{remark}
El coste de estimar el error de esta manera supone, aproximadamente, un incremento del 50\% en la cantidad de cómputo, comparándolo con solo calcular $y_h$.\\
Puede parecer un coste demasiado alto, pero generalmente merece la pera excepto para los problemas que más tiempo emplean. 
\end{remark}  
\begin{remark}
Este método también nos dice cuándo podemos coger un paso más grande y seguir manteniendo la tolerancia: si $q < \left(\frac{\varepsilon h}{\tilde{e}_r(h)}\right)^{\frac{1}{k}}$, podemos tomar $h=\left(\frac{\varepsilon h}{\tilde{e}_r(h)}\right)^{\frac{1}{k}}h$, que es más grande que el paso anterior, y se seguiría manteniendo $\tilde{e}_r(h)\leq \varepsilon\cdot h$.
\end{remark}

\subsection{Método de Runge-Kutta-Fehlberg}
Para ilustrar este método, supogamos que tenemos dos métodos de aproximación. El primero es un Runge-Kutta de orden $n$
$$
y(t_{i+1}) = y(t_i) + h\phi (t_i, y(t_i), h) + \mathcal{O}(h^{n+1})
$$
que produce las aproximaciones
$$
\left\{
\begin{array}{lll}
\omega_0 = \alpha & & \\
\omega_{i+1} = \omega_i + h\phi(t_i, \omega_i, h) & & i >0
\end{array}
\right.
$$
\\con error de truncamiento $\tau_{i+1}(h) = \mathcal{O}(h^n)$, y el segundo es otro método de Runge-Kutta pero de orden $n+1$
$$
y(t_{i+1}) = y(t_i) + h\tilde{\phi}(t_i, y(t_i), h) + \mathcal{O}(h^{n+2})
$$
que produce las aproximaciones
$$
\left\{
\begin{array}{lll}
\tilde{\omega_0} = a & & \\
\tilde{\omega}_{i+1} = \tilde{\omega}_i + h\tilde{\phi}(t_i, \tilde{\omega}_i, h) & & i >0
\end{array}
\right.
$$
\\con error de truncamiento $\tau_{i+1}(h) = \mathcal{O}(h^{n+1})$.\\
Suponiendo $\omega_i \approx y(t_i) \approx \tilde{\omega}_i$ y tomando un $h$ fijo tenemos que
$$
\begin{array}{l}
\tau_{i+1}(h) = \frac{y(t_{i+1}) - y(t_i)}{h} - \phi(t_i, y(t_i), h) \approx  \frac{y(t_{i+1}) - \omega_i}{h} - \phi(t_i, \omega_i, h) =\\
\\
 =\frac{y(t_{i+1}) - (\omega_i+ - h\phi(t_i, \omega_i, h))}{h} = \frac{1}{h}(y(t+i)-\omega_{i+1})
\end{array}
$$
Y, análogamente, se tiene que 
$$
\tilde{\tau}_{i+1}(h) \approx \frac{1}{h}(y(t+i)-\tilde{\omega}_{i+1})
$$
Si ahora sumamos y restamos el término $\frac{1}{h}\tilde{\omega}_{i+1}$ en la aproximación obtenida para $\tau_{i+1}(h)$, llegamos a que
$$
\tau_{i+1}(h) = \frac{1}{h}[(y(t_{i+1})-\tilde{\omega}_{i+1}) + (\tilde{\omega}_{i+1} - \omega_{i+1})] = \tilde{\tau}_{i+1}(h) + \frac{1}{h}(\tilde{\omega}_{i+1}-\omega_{i+1})
$$
Ahora, como $\tilde{\tau}_{i+1}(h)$ es una $\mathcal{O}(h^{n+1})$, podemos aproximar el error local de truncamiento del método de orden $n$ como
$$
\tau_{i+1}(h) \approx \frac{1}{h}(\tilde{\omega}_{i+1}-\omega_{i+1})
$$
\\De manera que la aproximación del error local de truncamiento al tomar el paso de tamaño $qh$ es
$$
\tau_{i+1}(qh)\approx C(qh)^n = Cq^nh^n \approx q^n \tau_{i+1}(h) \approx \frac{q^n}{h}(\tilde{\omega}_{i+1}-\omega_{i+1}) 
$$
donde hemos utilizado que $\tau_{i+1}(h) \approx Ch^n$ por ser $\tau_{i+1}(h)$ una $\mathcal{O}(h^n)$.
\\Para establecer la cota del error local de truncamiento por la tolerancia $\varepsilon$ tendríamos, por tanto, que escoger $q$ tal que
$$
q \leq \left(\frac{\varepsilon h}{|\tilde{\omega}_{i+1} - \omega_{i+1}|}\right)^{\frac{1}{n}}
$$
Un método muy usado que utiliza esta última desigualdad para controlar el error local de truncamiento es el \textbf{método de Runge-Kutta-Felhberg}, que pasamos a introducir a continuación.\\

Si eligiésemos dos métodos arbitrarios de Runge-Kutta de cuarto y quinto orden para aplicar lo que acabamos de ver, necesitaríamos, como mínimo, 10 evaluaciones de la función $f$ (4 por el método de orden 4 y 6 por el de orden 5). Sin embargo, \textbf{Erwin Fehlberg} encontró 2 métodos de Runge-Kutta de estos órdenes que se pueden anidar de manera que se necesiten tan solo 6 evaluaciones de $f$. \\
El \textbf{método de Runge-Kutta-Fehlberg} consiste, por tanto, en emplear Runge-Kutta con el error local de truncamiento de quinto orden
$$
\tilde{\omega}_{i+1} = \omega_i + \frac{16}{135}k_1 + \frac{6656}{12825}k_3 + \frac{28561}{56430}k_4 - \frac{9}{50}k_5 + \frac{2}{55}k_6
$$
para estimar el error local en un método de Runge-Kutta de cuarto orden dado por 
$$
\omega_{i+1} = \omega_i + \frac{25}{216}k_1 + \frac{1408}{2565}k_3 + \frac{2197}{4104}k_4 - \frac{1}{5}k_5 
$$
donde 
$$
\begin{array}{l}
\\
k_1 =  hf\left(t_i, \omega_i\right)\\
\\
k_2 =  hf\left(t_i + \frac{h}{4}, \omega_i + \frac{1}{4}k_1\right)\\
\\
k_3 =  hf\left(t_i + \frac{3h}{8}, \omega_i + \frac{3}{32}k_1 +  \frac{9}{32}k_2\right)\\
\\
k_4 = hf\left(t_i + \frac{12h}{13}, \omega_i + \frac{1932}{2197}k_1 -   \frac{7200}{2197}k_2 +  \frac{7296}{2197}k_3\right) \\
\\
k_5 = hf\left(t_i + h, \omega_i + \frac{439}{216}k_1 - 8k_2 +  \frac{3680}{513}k_3 - \frac{845}{4104}k_4\right) \\
\\
k_6 = hf\left(t_i + \frac{h}{2}, \omega_i - \frac{8}{27}k_1 + 2k_2 -  \frac{3544}{2565}k_3 + \frac{1859}{4104}k_4 - \frac{11}{40}k_5\right)
\end{array}
$$
\\ //aquí falta algoritmo completo
