\section{Métodos de paso fijo}

\begin{definition}  Un método es de paso fijo si la solución que genera cumple $t_i-t_{i-1}=h,\forall i=1,\dots n$ para cierto $h>0$ constante. \end{definition}

En este caso se tendrá $n=\lceil \frac{b-a}{2} \rceil$, y supondré este valor mientras trate con métodos de paso fijo.

\subsection{Método de Euler}
\begin{definition} 
    El método de Euler de paso fijo genera, dado el problema \ref{eqn:pvi} y un paso $h>0$:

\begin{equation} \label{eqn:eulersol}
\begin{cases}
    w_0=y_0 \\
    w_{i+1}=w_i + h\cdot f(t_i, w_i), \forall i=0,\dots, n-1
\end{cases}
\end{equation}

Siendo $t_i =t_0+i\cdot h,\forall i=0,\dots, n$.
\end{definition}

\subsubsection{Estimación del error con el método de Euler}

\begin{lemma}
\label{lema2}
\begin{enumerate}
    \item $\forall t \in \mathbb{R}$, $1+t \leq e^t$.
    \item Si $t\geq -1$, entonces $\forall m > 0$ se cumple que  $0\leq (1+t)^m \leq e^{mt}.$
\end{enumerate} 
\end{lemma}
\begin{proof}
\begin{enumerate}
    \item Utilizando el desarrollo de Taylor para la exponencial:
    $$
    e^t = 1+t+\frac{t^2}{2}e^{\xi} \textnormal{ con }\xi\in(0,t) \overset{\frac{t^2}{2}e^{\xi} \geq 0}{\implies} e^t \geq 1+t
    $$
    \item Como $t\geq -1$, $1+x \geq 0$, y entonces aplicando lo anterior tenemos que $$ 0\leq 1+t \leq e^t \overset{m\geq 0}{\implies} 0 \leq (1+t)^m \leq (e^t)^m = e^{tm} $$
\end{enumerate}
\end{proof}

\begin{theorem}[Convergencia del método de Euler]
    Sea $f:D\subset \mathbb{R}^2\rightarrow \mathbb{R}$ con $D$ abierto, e $Y(t)$ una solución de $(1)$ con $(t,Y(t))\in D,\forall t\in[a,b]$. Sea también la solución \ref{eqn:eulersol} para cierto $h$ fijo. Si se cumple
    \begin{enumerate}[label=(\alph*)]
        \item f Lipschitziana respecto de la segunda variable en $D$, con constante de Lipschitz $K$.
        \item $Y''$ existe en todo $[a,b]$ y está acotada por una constante $C\geq 0$.
        \item $(t_i,w_i)\in D,\forall i=0,\dots,n$.
    \end{enumerate}
    entonces $$\max_{0\leq i \leq n}|Y(t_i)-w_i| \leq e^{(b-a)K}\cdot |Y(a)-w_0| + \frac{e^{(b-a)K}-1}{2K}ch$$

\end{theorem}
\begin{proof}
    \begin{equation}
\label{eq1}
\left\{
\begin{array}{l}
Y(t_{n+1}) = Y(t_n) + hY'(t_n) + \frac{1}{2}h^2Y''(\xi) = Y(t_n) + hf(t_n, Y(t_n)) + \frac{1}{2}h^2Y''(\xi)\\
\\
y_h(t_{n+1}) = y_h(t_n) + hf(t_n, y_h(t_n))
\end{array}
\right. 
\end{equation}
\begin{equation}
\label{eq2}
\implies Y(t_{n+1}) - y_h(t_{n+1}) = (Y(t_n) - y_h(t_n)) + h(f(t_n, Y(t_n)) - f(t_n, y_h(t_n))) + \frac{1}{2}h^2Y''(\xi)
\end{equation}
De esta última expresión se deduce, llamando $error_i = \|Y(t_i) - y_h(t_i)\|$ para $0\leq i \leq N$ y teniendo en cuenta que $\left\|\frac{1}{2}h^2Y''(\xi)\right\| \leq \frac{1}{2}h^2c$:
$$
error_{n+1} = \|Y(t_{n+1}) - y_h(t_{n+1})\| \leq \|Y(t_n) - y_h(t_n)\| + h\overset{\leq k\| Y(t_n) - y_h(t_n)\|}{\overbrace{\|f(t_n, Y(t_n)) - f(t_n, y_h(t_n))\|}}  + \frac{1}{2}h^2c \leq
$$
$$
\leq (1+hk)\|Y(t_n) - y_h(t_n)\| + \frac{1}{2}h^2c = (1+hk)error_n + \frac{1}{2}h^2c
$$
Es decir, se obtiene que $error_{n+1} \leq (1+hk)error_n + \frac{1}{2}h^2c$.\\\\Si ahora utilizamos esta última expresión recursivamente como sigue
$$
error_{n+1} \leq (1+hk)error_n + \frac{1}{2}h^2c \leq (1+hk)\left[(1+hk)error_{n-1} + \frac{1}{2}h^2c\right] + \frac{1}{2}h^2c
$$
Llegamos a que
$$
error_{n+1} \leq (1+hk)error_0 + \overset{\textnormal{serie geométrica entre 0 y n}}{\overbrace{\left[ 1 + (1+hk) + (1+hk)^2 + \cdots + (1+hk)^{n}\right]}}\frac{1}{2}h^2c =  
$$
$$
= (1+hk)error_0 + \left[\frac{(1+hk)^n - 1}{hk}\right]\frac{1}{2}h^2c = (1+hk)error_0 + \left[\frac{(1+hk)^n - 1}{k}\right]\frac{1}{2}hc \overset{\ref{lema2}}{\leq} 
$$
$$
\leq e^{hkn}error_0 + \left[\frac{e^{hkn} - 1}{k}\right]\frac{1}{2}hc =  e^{(b-a)k}\|Y(t_0) - y_h(t_0)\| + \left[\frac{e^{(b-a)k} - 1}{k}\right]\frac{1}{2}hc
$$

\end{proof}

\begin{theorem}[Convergencia del método de Euler con error de redondeo]
    Sea $f:D\subset \mathbb{R}^2\rightarrow \mathbb{R}$ con $D$ abierto, e $Y(t)$ una solución de $(1)$ con $(t,Y(t))\in D,\forall t\in[a,b]$.

    Fijado $h>0$, sea la solución numérica

\begin{equation}
\begin{cases}
    w_0=y_0 \\
    w_{i+1}=w_i + h\cdot f(t_i, w_i) + \delta_i
\end{cases}
\end{equation}

con $\delta_i<\delta,\forall i=0,\dots,n$.

    Si se cumple
    \begin{enumerate}[label=(\alph*)]
        \item f Lipschitziana respecto de la segunda variable en $D$, con constante de Lipschitz $K$.
        \item $Y''$ existe en todo $[a,b]$ y está acotada por una constante $C\geq 0$.
        \item $(t_i,w_i)\in D,\forall i=0,\dots,n$.
    \end{enumerate}
    entonces $$\max_{0\leq i \leq n}|Y(t_i)-w_i| \leq e^{(b-a)K}\cdot |Y(a)-w_0| + \frac{e^{(b-a)K}-1}{K}\left(\frac{1}{2}ch+\frac{\delta}{h}\right)$$

\end{theorem}

Se puede estimar el error de otra forma bajo ciertas condiciones, que nos será muy útil para estimar numéricamente el error e incluse obtener soluciones mejores a partir del método de Euler.

\begin{theorem}
    Dado el PVI \ref{eqn:pvi} y la solución \ref{eqn:eulersol}, si la solución $Y(t)$ real es de clase $C^3$ y $\frac{\partial f}{\partial y},\frac{\partial^2f}{\partial y^2}$ constantes, entonces
    \begin{equation}
    Y(t_i)-w_i=h\cdot D(t_i)+\theta(h^2),\forall t\in[t_0,b]
    \end{equation}
    dónde $D$ es solución del siguiente PVI:
\begin{equation}
\begin{cases}
    d'(t)=g(t)d(t) + \frac{1}{2} Y''(t), g(t)=\frac{\partial f(t,y)}{\partial y}(t,Y(t)) \\
    d(a) = 0 \\
    t\in[a,b]
\end{cases}
\end{equation}
\end{theorem}
\begin{proof}
Puede encontrarse en el libro de Atkinson, pero no entra.
\end{proof}

\begin{remark}
Cuando $D(t)$ se puede obtener explícitamente, el término $hD(t_i)$ de la fórmula  (\ref{eq3}) normalmente proporciona una estimación bastante buena del error real cometido ($Y(t_i) - y_h(t_i)$), y la precisión de la estimación mejora si se disminuye el tamaño del paso $h$.
\end{remark}
 No suele ser muy práctico encontrar $D$, pero el resultado nos ayuda a demostrar lo siguiente:

\begin{theorem}
    En condiciones similares a las del teorema anterior, tomando $(t_i^h,w_i^h)$ y $(t_i^{h/2},w_i^{h/2})$ soluciones obtenidas con el método de Euler con pasos $h$ y $\frac{h}{2}$ respectivamente:
    \begin{enumerate}[label=(\alph*)]
        \item $Y(t_i^h)-w_{2i}^{h/2} = (w_{2i}^{h/2}-w_i^h) + \theta(h^2)$.
        \item $w_i := 2\cdot w_{2i}^{h/2}-w_i^h$ es una aproximación de $Y(t)$ de orden $\theta(h^2)$.
    \end{enumerate}
\end{theorem}

\begin{proof}
    Aplicando el teorema anterior obtenemos:
    $$
    Y(t_i^h)-w_i^h=h\cdot D(t_i^h)+\theta(h^2),\forall t\in[t_0,b]
    $$
    $$
    Y(t_{2i}^{h/2})-w_{2i}^{h/2}=h\cdot D(t_{2i}^{h/2})+\theta(h^2),\forall t\in[t_0,b]
    $$
    Basta observar que $t_i^h = t_{2i}^{h/2}$ y restar la primera ecuación a la segunda multiplicada por 2. Despejando se obtiene tanto (a) como (b).

\end{proof}

\subsection{Estimación general del error}

Como hemos visto, a partir de conocer el error cometido en cada paso hemos logrado estimar globalmente el error de la solución generada por el método de Euler. La idea es hacer esto de manera general, pero de momento nos conformamos con formalizar la idea de error local.

\begin{definition}
Dado un método de diferencias
\begin{equation} \label{eqn:diffmet}
\begin{cases}
    w_0=\alpha \\
    w_{i+1}=w_i + h\cdot \phi(t_i, w_i), \forall i=0,\dots, n-1
\end{cases}
\end{equation}
    que devuelva una solución de \ref{eqn:pvi} con solución real $Y$, se define su \textbf{error local de truncamiento (e.l.t.)} como
    $$
    T_{i+1}(h) = \frac{Y(t_{i+1})-Y(t_i)-h\phi(t_i,Y(t_i))}{h}
    = \frac{Y(t_{i+1})-Y(t_i)}{h} -h\phi(t_i,Y(t_i))
    $$
\end{definition}


\begin{example}
En el caso concreto del método de Euler se tiene:
    $$
    T_{i+1}(h)
    = \frac{Y(t_{i+1})-Y(t_i)}{h} -h\cdot f(t_i,Y(t_i))= \frac{h}{2} Y''(\xi_i),
    \xi_i\in[t_i,t_{i+1}], \forall i=0,\dots, n-1
    $$

    y si $|Y''(t)|\leq C,\forall t\in[a,b]$ entonces
    $$
    |T_{i+1}(h)|\leq \frac{C}{2} \cdot h
    $$
    que es $\theta(h)$.

Lo que nos interesa es que este error decrezca lo más rápido posible con h, es decir, que sea $\theta(h^p)$ con $p$ lo más grande posible. En esta búsqueda llegamos al siguiente método, el de Taylor.
\end{example}

\subsection{Método de Taylor}
 
De aquí en adelante asumiré el PVI general del principio, y cualquier método que considere intentará aproximar una solución suya.

\begin{definition}
Se define el método de Taylor de orden p como:
\begin{equation}
\begin{cases}
    w_0=\alpha \\
    w_{i+1}=w_i + h\cdot T^{(p)}(t_i, w_i), \forall i=0,\dots, n-1
\end{cases}
\end{equation}

    dónde $T^{(p)}(t_i,w_i)=\sum_{i=0}^{p-1} \frac{h^i}{(i+1)!}f^{(i)}(t_i,w_i)$, y $f^{(i)}(t_i,w_i)$ viene dado po $Y^{(i+1)}(t)=(f(t,Y(t)))^{(i)}$.
\end{definition}

Unas pocas cuentas son suficientes para darse cuenta de que calcular estas derivadas de manera genérica es realmente costoso, y normalmente se suelen hace las cuentas de manera \textit{adhoc} para un problema concreto.

El siguiente teorema era de esperar:

\begin{theorem}
    Si Y es de clase $C^{(p+1)}([a,b])$, el e.l.t. del método de Taylor de orden p es $\theta(h^p)$.
\end{theorem}
\begin{proof}
En estas condiciones el e.l.t. no es más que el resto de Lagrange:

$$
    T_{i+1}(h)=\frac{h^p}{(p+1)!} f^{(p)}(\xi_i,Y(\xi_i)),\xi_i\in [t_i,t_{i+1}]\subset[a,b]
$$
    Por ser $f(t,Y(t))$ contínua en compacto está acotada en $[a,b]$, y eso es por lo tanto $\theta(h^p)$.
\end{proof}

%TODO esto solo para una dimensión?

A pesar de este gran resultado respecto al error, la dificultad para aplicarlo de manera genérica y los problemas que conlleva la derivación numérica no lo hace muy práctico. Buscamos entonces métodos que no necesiten estas derivadas y que tengan órdenes similares.

\subsection{Métodos de Runge-Kutta}

Comenzamos buscando métodos de orden 2, para lo cual queremos aproximar $T^{(2)}(t,y)=f(t,y)+\frac{h}{2} f'(t,y)$ con un error de orden $\theta(h^2)$. Una idea feliz es considerar una aproximación de la forma $a_1f(t+\alpha_1, y+\beta_1)$. Aproximando esto último por Taylor tenemos:

$$
a_1f(t+\alpha_1, y+\beta_1)=a_1\left(
f(t,y)+
\frac{\partial f}{\partial t}(t,y)\cdot \alpha_1+
\frac{\partial f}{\partial y}(t,y)\cdot \beta_1+
\frac{\partial^2 f}{\partial t^2}(\xi,\mu)\cdot \frac{\alpha_1^2}{2}+
\frac{\partial^2 f}{\partial t \partial y}(\xi,\mu)\cdot \alpha_1\beta_1+
\frac{\partial^2 f}{\partial y^2}(\xi,\mu)\cdot \frac{\beta_1^2}{2}
\right)
$$

Nótese que suponemos suficiente regularidad en f para que las parciales cruzadas sean iguales. Haciendo $a_1=1,\alpha_1=\frac{h}{2}, \beta_1=\frac{h}{2} f(t,y)$ la igualdad anterior se traduce en:

$$
f(t+\frac{h}{2}, y+\beta_1)=
f(t,y)+
\frac{\partial f}{\partial t}(t,y)\cdot \frac{h}{2}+
\frac{\partial f}{\partial y}(t,y)\cdot \frac{h}{2} f(t,y)+
\frac{\partial^2 f}{\partial t^2}(\xi,\mu)\cdot \frac{h^2}{8}+
\frac{\partial^2 f}{\partial t \partial y}(\xi,\mu)\cdot \frac{h^2}{4}f(t,y)+
\frac{\partial^2 f}{\partial y^2}(\xi,\mu)\cdot \frac{h^2}{8}f(t,y)
$$

$$
=T^{(2)}(t,y)+
h^2 \left(
\frac{\partial^2 f}{\partial t^2}(\xi,\mu)\cdot \frac{1}{8}+
\frac{\partial^2 f}{\partial t \partial y}(\xi,\mu)\cdot \frac{1}{4}f(t,y)+
\frac{\partial^2 f}{\partial y^2}(\xi,\mu)\cdot \frac{1}{8}f(t,y)
\right)
$$

De nuevo bajo buenas condiciones de $f$ lo que hay entre paréntesis está acotado, y al multiplicar por $h^2$ tenemos el orden de error buscado. Esto nos da nuestro primer método de Runge-Kutta:

\begin{definition}[Método del Punto Medio]
\begin{equation}
\begin{cases}
    w_0=\alpha \\
    w_{i+1}=w_i + h\cdot f(t+\frac{h}{2}, y+\frac{h}{2}f(t,y)) , \forall i=0,\dots, n-1
\end{cases}
\end{equation}
\end{definition}

\begin{remark}
    En este método no necesitamos solo necesitamos la función $f$, pero necesitamos dos evaluaciones en cada paso, que puede ser costoso.
\end{remark}

\begin{remark}
    Al ser $ f(t+\frac{h}{2}, y+\beta_1)=T^{(2)}(t,y)+\theta(h^2) $, el e.l.t. será:
    $$
    T_{i+1}(h)= \left(\frac{Y(t_{i+1})-Y(t_i)-h\phi(t_i,Y(t_i))}{h} - T^{(2)}(t_i, y(t_i))\right) - \theta(h^2)
    =\theta(h^2)- \theta(h^2)
    =\theta(h^2)
    $$
\end{remark}

Podríamos intentar una estrategia parecida para $T^{(3)}(t,y)$:

\begin{equation}
\label{eq5}
T^{(3)}(t,y) = f(t,y) + \frac{h}{2}f'(t,y) + \frac{h^2}{6}f''(t,y) \approx a_1f(t,y) + a_2f(t+\alpha_2, y+\delta_2f(t,y))
\end{equation}
pero ni siquiera esto basta para igualar el término $ \frac{h^2}{6}\left[ \frac{\partial f}{\partial y}(t,y)\right]^2f(t,y)$, que resulta al desarrollar $\frac{h^2}{6}f''(t,y)$.
Obtenemos sin embargo más métodos de orden 2:

\begin{definition}[Método modificado de Euler]
Se obtiene al tomar $a_1 = a_2 = \frac{1}{2}$ y $\alpha_2 = \delta_2 = h$ en la ecuación  (\ref{eq5}).
\begin{equation}
\begin{cases}
    w_0=\alpha \\
    w_{i+1}=w_i + \frac{h}{2}\left[f(t_i, w_i) + f(t_{i+1}, w_i+hf(t_i,w_i))\right]  , \forall i=0,\dots, n-1
\end{cases}
\end{equation}

\end{definition}

\begin{definition}[Método de Heun]
Este método se obtiene al tomar $a_1 = \frac{1}{4}$, $a_2 = \frac{3}{4}$ y $\alpha_2 = \delta_2 = \frac{2}{3}h$ en la ecuación  (\ref{eq5}).
\begin{equation}
\begin{cases}
    w_0=\alpha \\
    w_{i+1}=w_i + \frac{h}{4}\left[f(t_i, w_i) + 3f(t_i + \frac{2}{3}h, w_i+\frac{2}{3}hf(t_i,w_i))\right] , \forall i=0,\dots, n-1
\end{cases}
\end{equation}

\end{definition}

El método más utilizado de Runge-Kutta es el siguiente, de orden 4:

\begin{definition}[Método de Runge-Kutta de orden 4]
\begin{equation}
\begin{cases}
    w_0=\alpha \\
    K_1^{i+1} = hf(t,w_i)\\
    K_2^{i+1} = hf(t + h/2,w_i+ K_1^{i+1}/2)\\
    K_3^{i+1} = hf(t + h/2,w_i+ K_2^{i+1}/2)\\
    K_4^{i+1} = hf(t + h,w_i + K_3^{i+1})\\

    w_{i+1}=w + \frac{K_1^{i+1} + 2K_2^{i+1} + 2K_3^{i+1} + K_4^{i+1}}{6}
\end{cases}
\end{equation}

\end{definition}


Es fácil ver que si queremos más precisión tenemos que utilizar más evaluaciones de $f$. Una pregunta interesante es el número mínimos de evaluaciones en cada paso para conseguir una aproximación de cierto orden. La siguiente tabla relaciona ambos conceptos:

\begin{table}[H]
\centering
\begin{tabular}{|l||l|l|l|l|}
    \hline
Evaluaciones & $1\leq p \leq 4$ & $5\leq p\leq 7$ & $8\leq p\leq 9$  & $10\leq p$ \\
    \hline
Mejor e.l.t. & $\theta(h^p)$ & $\theta(h^{p-1})$ & $\theta(h^{p-2})$ & $\theta(h^{p-3})$ \\
    \hline
\end{tabular}
\caption{Tabla de Butcher de evaluaciones y órdenes del e.l.t.}
\end{table}

Muchas veces utilizamos el número de evaluaciones de la función $f$ como medida de complejidad, pues puede se muy costosa.

Tenemos también dos formas de mejora la precisión de la solución numérica: reducir el paso y usar un método de orden mayor. Veamos la comparación de estas dos formas con los métodos de Runge-Kutta vistos hasta el momento, fijando el número de evaluaciones a $40$ y suponiendo un intervalo con $b-a=1$:

\begin{table}[H]
\centering
\begin{tabular}{|l|l|l|}
    \hline
Método           & Paso  & Precisión aproximada \\
    \hline
    \hline
RK4              & 0.1   & $10^{-4}$ \\
    \hline
Euler Modificado & 0.05  & $2.5\cdot 10^{-3}$ \\
    \hline
Euler            & 0.025 & $2.5\cdot 10^{-2}$ \\
    \hline
\end{tabular}
\end{table}

En general obtenemos mejores resultados aumentando el orden del método.


\section{Métodos de paso adaptativo}
