\section{Relaciones de recurrencia lineales homogéneas}

Para afrontar el resto del texto vamos a necesitar
conocer las relaciones de recurrencia lineales.
En particular, la estructura de sus soluciones.

\begin{definition}
    Una \emph{relación de recurrencia lineal homogénea} de orden $n$
    es una ecuación de la forma
    \begin{equation*}
        x_i = a_1x_{i-1} + a_2x_{i-2} + \cdots + a_nx_{i-n},
        \quad a_n \ne 0,
    \end{equation*}
    donde los coeficientes $a_1,\ldots,a_n$ son constantes,
    cuyas soluciones son las sucesiones $\{x_n\}_{n \in \N}$
    que verifican la ecuación para todo $i \ge n$.

    A las sucesiones unívocamente determinadas por
    $n$ valores iniciales $x_0,\ldots,x_{n-1}$
    y una relación de recurrencia de orden $n$
    se les llama \emph{sucesiones de recurrencia}.
\end{definition}

\begin{remark}
    La relación de la definición anterior es lineal porque el $i$-ésimo término
    es una combinación lineal de $1$ y los $n$ elementos anteriores
    y es homogénea porque el coeficiente que acompañaría al $1$,
    el término independiente,
    es $0$.
\end{remark}

\begin{theorem}
    El conjunto de soluciones de una
    relación de recurrencia lineal homogénea de orden $n$
    es un espacio vectorial de orden $n$.
\end{theorem}

\begin{proof}
    El conjunto de las sucesiones reales o complejas
    es un espacio vectorial de dimensión infinita.
    Sea
    \begin{equation*}
        x_i = a_1x_{i-1} + a_2x_{i-2} + \cdots + a_nx_{i-n},
        \quad a_n \ne 0,
    \end{equation*}
    una relación de recurrencia de orden $n$.
    Dadas dos soluciones $\{x_m\}_{m \in \N}$ e $\{y_m\}_{m \in \N}$
    y un escalar $\alpha \in \R$,
    tanto la suma como el producto
    \begin{gather*}
        \{x_m\}_{m \in \N} + \{y_m\}_{m \in \N} =
            \{x_m + y_m\}_{m \in \N} \qq{y} \\
        \alpha\{x_m\}_{m \in \N} =
            \{\alpha x_m\}_{m \in \N}
    \end{gather*}
    son soluciones de la relación también.
    Por tanto, el conjunto de soluciones es un subespacio vectorial.

    Para determinar la dimensión, basta con ver que las $n$ soluciones
    \begin{equation*}
        x^i_m =
        \begin{cases}
            0 & \text{si $i \ne m < n$} \\
            1 & \text{si $i = m$} \\
            a_1x^i_{m-1} + a_2x^i_{m-2} + \cdots + a_nx^i_{m-n}
                & \text{si $m \ge n$.}
        \end{cases}
    \end{equation*}
    son linealmente independientes y generan cualquier otra solución
    \begin{equation*}
        \{x_m\}_{m \in \N} =
        x_0\{x^0_m\}_{m \in \N} + x_1\{x^1_m\}_{m \in \N} + \cdots +
        x_{m-1}\{x^{m-1}_m\}_{m \in \N}. \qedhere
    \end{equation*}
\end{proof}

\subsection{Expresión general de una solución}

Si intentamos ver cuándo una sucesión geométrica $\{z^m\}_{m \in \N}$
es solución de una relación de recurrencia lineal homogénea,
obtenemos una ecuación de la forma
\begin{equation*}
    z^i = a_1z^{i-1} + a_2z^{i-2} + \cdots + a_nz^{i-n} \iff
    z^n = a_1z^{n-1} + a_2z^{n-2} + \cdots + a_n.
\end{equation*}
Por tanto, las soluciones de la segunda ecuación,
que son las raíces de
\begin{equation*}
    z^n - a_1z^{n-1} - a_2z^{n-2} - \cdots - a_n,
\end{equation*}
dan lugar a sucesiones que cumplen la ecuación lineal homogénea.

\begin{definition}
    El \emph{polinomio característico} de 
    una relación de recurrencia lineal homogénea
    es el polinomio
    \begin{equation*}
        z^n - a_1z^{n-1} - a_2z^{n-2} - \cdots - a_n,
    \end{equation*}
    donde $a_i$, $i = 1,\ldots,n$, son los coeficientes de la relación.
\end{definition}

\begin{remark}
    El $0$ no puede estar entre las soluciones del polinomio característico de
    una relación de recurrencia lineal homogénea porque $a_n \ne 0$.
\end{remark}

\begin{proposition}
    Si el polinomio caracerístico de
    una relación de recurrencia lineal homogénea de orden $n$
    tiene $n$ raíces distintas $\lambda_1,\ldots,\lambda_n$,
    entonces las soluciones de la relación son precisamente las sucesiones
    \begin{equation*}
        x_m = \sum_{i=1}^n c_i\lambda_i^m
    \end{equation*}
    con $c_i \in \R$, $i = 1,\ldots,n$,
    unas constantes que determinan la solución.
\end{proposition}


\begin{proof}
    Sabemos que las $n$ sucesiones $\{\lambda_i^m\}_{m \in \N}$
    son soluciones de la relación.
    Si además son linealmente independientes,
    habremos demostrado el enunciado.
    Si suponemos que no son linealmente independientes entonces
    existe una combinación lineal no nula
    \begin{equation*}
        \sum_{i=0}^n \alpha_i \{\lambda_i^m\}_{m \in \N} = 0 \iff
        \sum_{i=0}^n \alpha_i \lambda_i^m = 0
            \quad \text{para todo $m \in \N$.} \iff
        \sum_{i=0}^n \alpha_i \lambda_i^m = 0 \qq{para todo $m < n$.}
    \end{equation*}
    Las últimas $n$ ecuaciones lineal forman el sistema lineal
    \begin{equation*}
        \begin{pmatrix}
            1 & 1 & \cdots & 1 \\
            \lambda_1 & \lambda_2 & \cdots & \lambda_n \\
            \lambda_1^2 & \lambda_2^2 & \cdots & \lambda_n^2 \\
            \vdots & \vdots & \ddots & \vdots \\
            \lambda_1^{n-1} & \lambda_2^{n-1} & \cdots & \lambda_n^{n-1}
        \end{pmatrix}
        \begin{pmatrix}
            \alpha_1 \\ \alpha_2 \\ \alpha_3 \\ \vdots \\ \alpha_n
        \end{pmatrix}
        = 0.
    \end{equation*}
    La matriz del sistema es la traspuesta de una matriz de Vandermonde.
    Por tanto, en el caso en el que $\lambda_i \ne \lambda_j$ para $i \ne j$,
    que es el que nos atañe,
    es de rango máximo y la única solución al sistema es la trivial,
    lo que termina la prueba.
\end{proof}

\begin{remark}
    En la práctica, podemos determinar los coeficientes $c_i$
    dados los primeros $n$ términos de la sucesión de recurrencia
    resolviendo un sistema de ecuaciones lineales.
\end{remark}

\begin{example}
    Consideremos la sucesión de recurrencia
    $x_n = \frac{13}{3}x_{n-1} - \frac{4}{3}x_{n-2}$
    con condiciones iniciales $x_0 = 1$, $x_1 = \frac{1}{3}$.

    Su polinomio característico es $r^2-\frac{13}{3}r+\frac{4}{3} = 0$
    de raíces $\lambda_1 = 4$ y $\lambda_2 = \frac{1}{3}$
    Por tanto, nuestra sucesión recurrente es de la forma
    $x_n = a4^n + b\frac{1}{3^n}$.
    Resolviendo el sistema
    \begin{equation*}
        \left\{\begin{matrix}
        x_0 = a+b = 1 \\
        x_1 = 4a + \frac{b}{3} = \frac{1}{3}
        \end{matrix}\right.
    \end{equation*}
    obtenemos que $a = 0$ y $b = 1$,
    por lo que $x_n = \frac{1}{3^n}$.
\end{example}

También tenemos un resultado para el caso en el que
el polinomio caracerístico no tiene $n$ raíces distintas.
Invitamos al lector a demostrarlo.

\begin{proposition}
    \newcommand{\mult}[1]{\operatorname{mult}(#1)}

    Si $\lambda_1,\ldots,\lambda_n$ son las raíces
    del polinomio caracerístico de
    una relación de recurrencia lineal homogénea de orden $n$
    entonces las soluciones de la relación son precisamente las sucesiones
    \begin{equation*}
        x_m =
        \sum_{\mult{\lambda_i} \ge 1}^n c_{j(1, i)}\lambda_i^m +
        \sum_{\mult{\lambda_i} \ge 2}^n c_{j(2, i)}n\lambda_i^m +
        \cdots +
        \sum_{\mult{\lambda_i} \ge n}^n c_{j(n, i)}n^{n-1}\lambda_i^m,
    \end{equation*}
    con $c_j \in \R$, $n$ constantes que determinan la solución.
\end{proposition}

