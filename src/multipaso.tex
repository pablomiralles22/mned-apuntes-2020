\section{Métodos multipaso}

Hasta ahora, al implementar un método de resolución numérica sólo usábamos
$\vw_i$ para calcular $\vw_{i+1}$,
ignorando los valores obtenidos anteriormente.
A pesar de que esto parezca obvio pues si $\vf$ es Lipschitziana
la solución está unívocamente determinada por un punto,
cuando buscamos una solución computacional
los valores pasados pueden ser de utilidad para ahorra cálculos.
Los métodos multipaso se construyen, una vez obtenidos los primeros $m$ pasos,
utilizando los $m$ valores anteriores para calcular el valor $\vw_{i+1}$.

\begin{definition}
    Un \emph{método en diferencias a $m$ pasos} es un método de paso fijo $h$
    que a partir de $m$ estados
    $\{(t_0 + jh, \vw_j) \approx (t_0 + jh, \vy(t_j))\}_{j = i-m+1,\ldots,i-1}$,
    aproxima $\vy(t_i) = \vy(t_0 + ih)$ como
    \begin{equation*}
        \vw_i = a_0\vw_{i-m} + a_1\vw_{i-m+1} + \dots + a_{i-1}\vw_{i-1}
            + h\vphi(h, t_i, \vw_{i-m}, \vw_{i-m+1}, \dots, \vw_{i-1}, \vw_i).
    \end{equation*}
\end{definition}

Ahora tenemos dos tipos de métodos, pues si $\vphi$ depende de $\vw_i$
será um método implícito y en otro caso será explicito.
En los implícitos tendremos que resolver una ecuación
para calcular el próximo punto.
Dos ejemplos de estos métodos son
el método explícito de Adams-Bashforth de $4$ pasos y
el método implícito de Adams-Moulton de $3$ pasos.
Nótese que si ahora tuvieramos $m = 1$,
tendríamos un método a un paso como hasta ahora.

\begin{method}{Método explícito de Adams-Bashforth de orden $4$}
    \label{met:AB4steps}

    El \emph{método de Adams-Bashforth} de orden $4$
    es un método a $4$ pasos.
    \begin{equation}
        \vw_{i+1} = \vw_i + \frac{h}{24}\qty\bigg[
            55\vf(t_i, \vw_i) - 59\vf(t_{i-1}, \vw_{i-1})
            + 37\vf(t_{i-2}, \vw_{i-2}) - 9\vf(t_{i-3}, \vw_{i-3})
        ].
    \end{equation}
\end{method}

\begin{proposition}
    El método de Adams-Bashforth de orden $4$
    tiene un error local de truncamiento
    de orden $\ELT_{i+1}(h) = \frac{251}{720}\vy^5(\xi)h^4 + \vo(h^5)$.
\end{proposition}

\begin{method}{Método implícito de Adams-Moulton de orden $4$}
    \label{met:AM3steps}

    El \emph{método de Adams-Moulton} de orden $4$
    es un método a $3$ pasos.
    \begin{equation}
        \vw_{i+1} = \vw_i + \frac{h}{24}\qty\bigg[
            9\vf(t_{i+1}, \vw_{i+1}) + 19\vf(t_i, \vw_i)
            - 5\vf(t_{i-1}, \vw_{i-1}) + \vf(t_{i-2}, \vw_{i-2})
        ].
    \end{equation}
\end{method}

\begin{proposition}
    El método de Adams-Bashforth de orden $4$
    tiene un error local de truncamiento
    de orden $\ELT_{i+1}(h) = \frac{19}{720}\vy^5(\xi)h^4 + \vo(h^5)$.
\end{proposition}

Podemos ver que ambos métodos requieren de una única evaluación de $f$
en cada paso y que el \cref{met:AM3steps} tiene un menor ELT pero a cambio
requiere alguna forma de despejar $\vw_{i+1}$.

\begin{remark}
    Resolver la ecuación implicita para $\vw_{i+1}$ no se puede hacer,
    en general, de forma analítica.
\end{remark}

\begin{remark}
    Existen métodos de Adams de diferentes órdenes y se obtienen de
    \begin{equation*}
        \vy'(t) = \vf(t,\vy(t)) \implies
        \vy(t_{i+1}) - \vy(t_i) =
        \int\limits_{t_i}^{t_{i+1}} \vy'(t) \dd{t} =
        \int\limits_{t_i}^{t_{i+1}} \vf(t,\vy(t)) \dd{t}.
    \end{equation*}
    Si quisiésemos aproximar $\vf(t,\vy(t))$ usando un polinomio interpolador
    $\vb{P}(t)$ en los puntos $t_{i-m+1},\dots,t_i$,
    tendríamos
    \begin{equation*}
        \vy(t_{i+1}) \approx
        \vy(t_i) + \int\limits_{t_i}^{t_{i+1}} \vb{P}(t) \dd{t}.
    \end{equation*}
    Por otra parte, los métodos implícitos usan el nodo $(t_{i+1}, \vw_{i+1})$
    a la hora de aproximar la integral.
\end{remark}

\begin{definition}
    Un \emph{método multipaso estándar a $m$ pasos}
    es un método multipaso en diferencias cuya expresión es
    \begin{equation*}
        \vw_i = a_0\vw_{i-m} + \dots + a_{m-1}\vw_{i-1} + h[
            b_0\vf(t_{i-m},\vw_{i-m}) + \dots + b_{m-1}\vf(t_{i-1},\vw_{i-1})
            + b_m\vf(t_i,\vw_i)
        ],
    \end{equation*}
    Si $b_m = 0$ entonces el método es explícito, en otro caso es implícito.
\end{definition}

Para tener disponibles los $m$ primeros $\vw_i$ para un método multipaso,
normalmente se utilizar un método de paso fijo.

