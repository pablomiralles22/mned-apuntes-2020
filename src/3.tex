\section{Métodos multipaso}

Hasta ahora, al implementar un método de resolución numérica sólo usábamos
$w_i$ para calcular los $w_{i+1}$ \eqref{eqn:diffmet}
ignorando los valores obtenidos anteriormente.
A pesar de que esto parezca obvio pues un p.v.i \eqref{eqn:pvi}
depende únicamente (si $f$ es Lipschitziana) de la condición inicial dada.
Sin embargo, esto solo es cierto si buscamos la solución real del problema,
cuando buscamos una solución computacional
los valores pasados pueden ser de utilidad.
Estos métodos multipaso se construyen, una vez obtenidos los primeros $m$ pasos,
utilizando los $m$ valores anteriores para calcular el valor $w_{i+1}$.
\begin{equation*}
    w_{i+1} = a_0w_{i-m+1} + a_1w_{i-m+2} + \dots + a_{m-1}w_i
        + h_iF(t_i,w_{i-m+1},w{i-m+2},\dots,w_i).
\end{equation*}
Nótese que si ahora tuvieramos $m = 1$,
tendríamos un método a un paso como hasta ahora.
Además ahora tenemos dos tipos de métodos multipaso, pues si usamos $w_{i+1}$
será um método implícito mientras que si no lo usamos será explicito.
Dos ejemplos de estos métodos son
el método explícito de Adams-Bashforth de $4$ pasos
y el método implícito de Adams-Moulton de $3$ pasos.

\begin{method}{Método explícito de Adams-Bashforth de orden $4$}
    El \emph{método de Adams-Bashforth} de orden $4$
    es un método a $4$-pasos.
    \begin{equation}\label{eqn:AB4steps}
        w_{i+1} = w_i + \frac{h}{24}\qty\bigg[
            55f(t_i,w_i) - 59f(t_{i-1},w_{i-1})
            + 37f(t_{i-1}w_{i-1}) - 9f(t_{i-3},w_{i-3})
        ].
    \end{equation}
\end{method}

\begin{proposition}
    El método de Adams-Bashforth de orden $4$
    tiene un error local de truncamiento
    de orden $T_{i+1}(h) = \frac{251}{720}y^5(\xi)h^4$.
\end{proposition}

\begin{method}{Método implícito de Adams-Moulton de orden $4$}
    El \emph{método de Adams-Moulton} de orden $4$
    es un método a $3$-pasos.
    \begin{equation}\label{eqn:AM3steps}
        w_{i+1} = w_i + \frac{h}{24}\qty\bigg[
            9f(t_{i+1},w_{i+1}) + 19f(t_i,w_i)
            - 5f(t_{i-2}w_{i-2}) + f(t_{i-2},w_{i-2})
        ].
    \end{equation}
\end{method}

\begin{proposition}
    El método de Adams-Bashforth de orden $4$
    tiene un error local de truncamiento
    de orden $T_{i+1}(h) = \frac{19}{720}y^5(\xi)h^4$.
\end{proposition}

Podemos ver que ambos métodos requieren de una única evaluación de $f$
en cada paso y que A-M \ref{eqn:AM3steps} tiene un menor e.l.t pero a cambio
requiere alguna forma de despejar $w_{i+1}$.

\begin{remark}
    Estos métodos de Adams se obtienen de lo siguiente
    \begin{equation*}
        y'(t) = f(t,y(t)) \implies
        y(t_{n+1}) - y(t_n) =
        \int\limits_{t_n}^{t_{n+1}} y'(t)dt =
        \int\limits_{t_n}^{t_{n+1}} f(t,y(t)) \dd t
    \end{equation*}
    Si ahora aproximamos $f(t,y(t))$ usando un polinomio interpolador $P(t)$
    en los puntos $t_{i-m+1},\dots,t_i$, tenemos
    \begin{equation*}
        y(t_{n+1}) \approx y(t_n) + \int\limits_{t_n}^{t_{n+1}} P(t)dt
    \end{equation*}
    Por otra parte, los métodos implícitos usan el nodo $(t_{i+1}w_{i+1})$
    a la hora de aproximar la integral.
\end{remark}

En general, los métodos multipaso son de la forma
\begin{equation*}
    w_{i+1} = a_0w_{i-m+1} + \dots + a_{m-1}w_i + h[
        b_0f(t_{i+1-m},w_{i+1-m}) + \dots + b_{m-1}f(t_i,w_i)
        + b_mf(t_{i+1},w_{i+1})
    ],
\end{equation*}
donde los $m$ primeros $w_j$ han sido calculados con otro método de un paso.
Si $b_m = 0$ entonces el método es explícito, en otro caso es implícito.
Resolver la ecuación implicita para $w_{i+1}$ no se puede hacer, en general,
de forma analítica.

\section{Teoría general de Convergencia}

Con los métodos multipaso hay que tener cuidado pues usamos
sucesiones recurrentes de orden mayor que uno,
lo que nos puede dar problemas por estar mal condicionada
(a parte de los problemas generados por el propio método).
Por ello introducimos las siguientes definiciones:

\begin{definition}
    Un método se dice \emph{consistente} (o \emph{compatible})
    con el p.v.i. que aproxima si
    \begin{equation*}
        \lim\nolimits_{h\to 0} \max{\norm{T_{i+1}(h)}} = 0.
	\end{equation*}
	El método se dice de orden $p$, $p \geq 1$, si $p$ es el mayor entero
	tal que $\max{\norm*{T_{i+1}(h)}} = \theta(h^p)$.
\end{definition}

Naturalmente solo consideraremos los métodos consistentes y de orden tan
alto como podamos encontrar teniendo un coste razonable.

\begin{definition}
    %TODO método en diferencias es multipaso o qué?
    % Hay un lío de términos importante
    Un método en diferencias
    \begin{equation*}
        w_{i+1} = \sum_{j=1}^m {a_{j-1}w_{i-m+j}}
            + hF(t_i,w_{i-m+j}, \dots, w_i,w_{i+1},h)
    \end{equation*}
    se dice convergente si
    \begin{equation*}
        \lim\nolimits_{h \to 0} \max{\norm{Y(t_i)-w_i}} = 0.
    \end{equation*}
\end{definition}

Un método convergente nos resuelve el problema en caso de una
implementacion de ``aritmética infinita''.
Lamentablemente solo disponemos de ordenadores de aritmética finita,
luego nuestros métodos no son los deseados sino los aproximados. %TODO EIN?
Por eso, introducimos la noción de método estable.

\begin{definition}
    Un método en diferencias se dice estable
    si al generar una segunda solución aproximada de la forma
    \begin{equation*}
        \tw_{i+1} = \sum_{j=1}^m {a_{j-1}\tw_{i-m+j}}
            + h F(t_i,\tw_{i-m+j}, \dots, \tw_i,\tw_{i+1},h) + \epsilon'_i,
    \end{equation*}
    existe una constante M, independiente de h, tal que
    \begin{equation*} \label{stab}
        \max_{m \leq n \leq N}{\norm{\tw_n - w_n}} \leq M[
            \max_{0 \leq i \ m-1}{\norm{\tw_i - w_i}}
            + \sum_{m-1 \leq j \leq N} \norm{\epsilon'_j}
        ]
    \end{equation*}
\end{definition}

\begin{remark}
    %TODO una referencia estaría increible en vd
    Esto no contradice el resultado que obtuvimos para Euler, pues
    %TODO esto pq?
    $\sum\norm{\epsilon_j} \approx n\epsilon \approx \frac{b-a}{h}\epsilon$.
\end{remark}

\begin{theorem}
    Si un método en diferencias
    \begin{equation*}
        w_{i+1} = \sum_{j=1}^m {a_{j-1}w_{i-m+j}}
            + hF(t_i,w_{i-m+j}, \dots, w_i,w_{i+1},h)
    \end{equation*}
    es estable y consistente, entonces es convergente.
    Si, además, es de orden $p \geq 1$, se tiene
    \begin{equation*}
        \max_{0 \leq n \leq N}\norm{x(t_n) - w_n} \leq M\qty\bigg[
            \max_{0 \leq i \leq m-1} \norm{x(t_i) - w_i} + Kh
            ]
    \end{equation*}
    Para constantes adecuadas M y K.
\end{theorem}

\begin{proof}
    Sea $\epsilon_i = hT_i(h_i)$ y utilicemos la estabilidad
    con $\tilde{w}_n = Y(t_n)$. Se tiene que
    \begin{equation*}
        \max_{0 \leq n \leq N}\norm{Y(t_n) - w_n} \leq M\qty\bigg[
            \max_{0 \leq i \leq m-1}\norm{Y(t_i) - w_i}
            + \sum_{m-1\leq n \leq N}\norm{\epsilon_n}
        ]
    \end{equation*}
    para algún $m$. Pero
    \begin{equation*}
        \sum_{m \leq n \leq N} \norm{\epsilon_n} =
        \sum_{m \leq n \leq N} h\norm{Tn+1(h)} \leq
        \qty(\sum_{m \leq n \leq N} h)
            \max_{m \leq n \leq N}\norm{T_{n+1}(h)} \leq
        (b-a)\max_{m \leq n \leq N}\norm{T_{n+1}(h)},
    \end{equation*}
    donde esta última tiende a $0$ cunado $h$ tiende a $0$
    por la condición de consistencia.
    Si además $\max\norm{T_{n+1}(h)} = O(h^p)$,
    $\max\norm{T_i+1I(h)} \leq \frac{kh^p}{b-a}$ para algún $K$
    de ahí el resultado.
\end{proof}

%TODO este teorema está fatal hay que mirarlo en el burden
\begin{theorem}
    Sea $w_{i+1} = w_i + h\phi(t, w_i, h)$ un método a un paso tal que
    $\phi(t_i, w_i, h)$ es continua y lipschitziana en $w_i$.
    Entonces existe $h_0 > 0$ tal que para todo $h < h_0$,
    \begin{enumerate}
        \item El método es estable
        \item El método es convergente si es consistente,
        lo que ocurre si $\phi(t, y, 0) = f(t, y)$ para todo $t$.
    \end{enumerate}
\end{theorem}

\begin{proof}
    Sea $L$ la constante de lipschitz de $\phi$ y
    sean $w_{i+1} = w_i + h\phi(t, w_i, h)$ y
    $\tw_{i+1} = h\phi(t, \tw_i, h) + \epsilon_i$.
    Entonces
    \begin{align*}
        \norm{\tw_{i+1} - w_{i+1}} & {} \le
            (1 + hL)\norm{\tw_i - w_i} + \norm{\epsilon_i} \\
        & \le (1 + hL)^2\norm{\tw_{i-1} - w_{i-1}}
            + (1 + hL)\norm{\epsilon_{i-1}} + \norm{\epsilon_i} \le \ldots \\
        \ldots & \le (1 + hL)^{i+1}\norm{\tw_0 - w_0} + \qty[
            \norm{\epsilon_i} + (1 + hL)\norm{\epsilon_{i-1}} + \dots
            + (1 + hL)^i\norm{\epsilon_0}
        ] \\
        & \le (1 + hL)^i\qty(
            \norm{\tw_0 - w_0} + \sum_{j=0}^i \norm{\epsilon_j}
        ).
    \end{align*}
    Lo que demuestra que el método es estable.

    Estudiamos ahora el error de truncamiento.
    \begin{multline*}
        Z_i(h) = \frac{y(t_i + h) - y(t)}{h} - \phi(t, y(t_i), h) =
        y'(\xi_i) - \phi(t, y(t_i), h) = \\
        f(\xi_i, y(\xi_i)) - \phi(t, y(t_i), h) =
        f(\xi_i, y(\xi_i)) - \phi(\xi_i, y(\xi_i), 0)
            + \phi(\xi_i, y(\xi_i), 0) - \phi(t, y(t_i), h).
    \end{multline*}
    Por la continuidad uniforme de la función
    $(t, h) \mapsto \phi(t, y(t), h)$
    sobre $[a, b] \times [0, h_0]$,
    dado $\epsilon > 0$ existe $h_0 > 0$ tal que si $h < h_0$,
    entonces como $\abs{\delta_i - t_i} < h$,
    \begin{equation*}
        \norm{\phi(\xi_i, y(\xi_i), 0) - \phi(t, y(t_i), h)} < \epsilon.
    \end{equation*}
    Por tanto, $\lim_{h \to 0} Z_i(h) = 0$
    si $f(t_i, y(t_i)) = \phi(t_i, y(t_i), 0)$.
\end{proof}

\subsection{Estudio de la estabilidad para métodos multipaso}

Estudiar la estabilidad de los métodos a $m$ pasos es más complicado.
En realidad, para estudiar la estabilidad basta hacerlo con el problema
$y' = 0$, $y(0) = \alpha$,
pero la demostración de este hecho no forma parte de estos apuntes.

Dado un método multipaso estándar a $m$ pasos,
\begin{equation*}
    w_i = a_0w_{i-m} + \dots + a_{m-1}w_{m-1} + h\qty\bigg[
        b_0f(t_{i-m}, w_{i-m}) + \dots + b_{m-1}f(t_{i-1}, w_{i-1})
        + b_mf(t_i, w_i)
    ]
\end{equation*}
el siguiente punto para el problema de prueba con $y' = 0$ se calcularía
siguiendo la ecuación de recurrencia
\begin{equation*}
    w_i = a_0w_{i-m} + \dots + a_{m-1}w_{m-1} + a_mw_m,
\end{equation*}
cuyo polinomio característico tiene raíces que satisfacen
\begin{equation*}
    z^m = a_0 + \dots + a_{m-1}z^{m-1}.
\end{equation*}
Si todas las raíces reales, $\lambda_1,\dots,\lambda_n$, fuesen distintas,
la ecuación general sería
\begin{equation*}
    w_n = \sum_{j=1}^m c_j\lambda_j^n
\end{equation*}
con los $c_j$ en función de $m$ puntos iniciales.

Lo primero que podemos afirmar es que como $y(t) = \alpha$ es una solución,
si el método tiene error local de truncamiento $\theta(h^p)$, $p \ge 1$,
$w_n = \alpha$ debe ser una posible solución:
\begin{equation*}
    \alpha = a_0 + a_1\alpha + \dots + a_{m-1}\alpha.
\end{equation*}
Es decir, $\lambda = 1$ es una de las raíces del polinomio caracerístico.
por lo que las soluciones son de la forma
\begin{equation*}
    w_n = c_1\alpha + \sum_{j=2}^m c_j\lambda_j^n,
\end{equation*}
y para nuestra ecuación de prueba sería $c_j = 0$ para todo $j$.

Para estudiar la estabilidad,
nos interesa que acotando los errores de redondeo en las condiciones iniciales
se acoten también los errores en la solución calculada.
Eso motiva la siguiente definición.

\begin{definition}
    Sean $\lambda_1,\ldots,\lambda_n$ las $n$ raíces del
    polinomio característico
    \begin{equation*}
        \lambda^m - a_{m-1}z^{m-1} - a_{m-2}z^{m-2} - \dots - a_0
    \end{equation*}
    de un método a $m$ pasos
    \begin{equation}\label{eqn:multistep}
        w_{i+1} = a_0w_{i-m+1} + a_1w{i-m+2} + \dots + a_{m-1}w_i
            + h_iF(t_i,w_{i-m+1},w{i-m+2},\dots,w_i).
    \end{equation}
    El método cumple la \emph{condición de raíz} si
    $\abs{\lambda_i} \le 1$ para todo $i = 1,\ldots, m$
    %TODO este tipo de añadidos que son necesarios si no asumimos todas distintas
    % hay que ver si los quitamos porque con Esquembre no lo hemos visto
    % pero especificamos que lo hemos quitado
    % o si lo dejamos y lo marcamos con otro color por ejemplo
    % o si lo quitamos
    y todas las raíces de valor absoluto uno son simples.
\end{definition}

\begin{theorem}
    Un método multipaso es estable si y solo si cumple la condición de raíz.
    Además, si el método es consistente,
    que sea estable es equivalente a que sea convergente.
\end{theorem}

\section{Implementación de los métodos de Adams. Predictor-corrector}

Estos métodos sirven para obtener lo mejor de los métodos explícito e implícito,
la completitud del primero y el mejor error del segundo,y se consiguen usando
el implícito para corregir la predicción del método explícito. Por ejemplo,
uno de estos métodos  sería usar como predictor el método de Adams-Bashforth 
de $4$ pasos (\cref{eqn:AB4steps}) y como corrector el método de Adams-Moulton
de $3$ pasos (\cref{eqn:AM3steps}).

De manera parecida a RKF, %TODO referencia a RKF
el método predictor-corrector obtiene, cada vez,
dos aproximaciones de la solución,
una obtenida de uno del predictor y otra del corrector.
Por la definición de error local de truncamiento tenemos que
\begin{gather*}
    y(t_{i+1}) = y(t_i) + \frac{h}{24}[
        55f(t_i,y(t_i) - 59f(t_{i-1},y(t_{i-1})) + 37f(t_{i-2}, y(t_{i-2}))
        - 9f(t_{w-3},y(t_{i-3})))
    ] + \frac{251}{720}y^{(5)}(\xi)h^5  \\
    y(t_{i+1}) = y(t_i) + \frac{h}{24}[
        9f(t_{w+1},y(t_{i+1})) + 19f(t_i,y(t_i) - 5f(t_{i-1},y(t_{i-1}))
        + f(t_{i-2}, y(t_{i-2})))
    ] - \frac{19}{720}y^{(5)}(\mu)h^5
\end{gather*}
Y si ahora identificamos los $w_j \approx y(t_j)$ para $j = 0,1,\dots, i$,
tenemos que
\begin{equation*} 
    y(t_{i+1} - w_{i+1}^{(0)}) \approx {} 
        \frac{251}{720}y^{(5)}(\xi)h^5
\end{equation*}
\begin{equation} \label{eqn:pred-corr}
    y(t_{i+1} - w_{i+1}^{(1)}) \approx {} 
        \frac{-19}{720}y^{(5)}(\mu)h^5.
\end{equation}
Si además, $h$ es suficientemente pequeño entonces
$y^{(5)}(\xi) \approx y^{(5)}(\mu)$
y restando
\begin{align*}
    w_{i+1}^{(1)} - w_{i+1}^{(0)} \approx 
        \frac{h^6}{720}\qty[251y^{(5)}(\xi) + 19y^{(5)}(\mu)] \approx &
        \frac{27}{72}h^5y^{(5)}(\mu) = \frac{3}{8}h^5y^(\mu) \\
    y^{(5)}(\mu) = \frac{8}{3h^5}(w_{i+1}^{(1)} - w_{i+1}^{(0)}) &
\end{align*}
Por lo tanto, de \ref{eqn:pred-corr} tenemos
\begin{equation*}
    y(t_{i+1}) - w_{i+1}^{(1)} \approx \frac{-19 * 8}{720 * 3}(w_{i+1}^{(1)} - w_{i+1}^{(0)})
\end{equation*}
Es decir
\begin{equation*}
    \tilde{e}_r(h) = \frac{19}{270}\norm{w_{i+1}^{(0)} - w_{i+1}^{(1)}}   
\end{equation*}
es una aproximación del error para un método de orden 4.
Luego dada una tolerancia $\epsilon > 0$, usando la estimación
local del paso del hijo de Ricardo tenemos
\begin{equation*}
    q = \bigg(\frac{\epsilon h}{\tilde{e}_r(h)}\bigg)^{1/4} \approx 
        2.48\bigg(\frac{\epsilon h}{\norm{w_{i+1}^{(0)} - w_{i+1}^{(1)}}}\bigg)^{1/4}
\end{equation*}
%TODO dice algo de ser mas consevador que blablabla
Por otra parte, cambiar el paso supone reiniciar los
$4$ primeros pasos de modo que se ignora el cambio de paso 
si tenemos que 
\begin{equation*}
    \frac{\epsilon h}{10} < \tilde{e}(h)
\end{equation*}