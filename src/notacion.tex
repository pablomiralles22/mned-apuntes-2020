\section{Notación}\label{sec:notation}

\begin{description}
    \item[$\vf : \Omega \subset \R \times \R^n \to \R^n$] es la función
    derivada de la expresión de un PVI general
    \begin{equation*}
        \left\{
        \begin{aligned}
            \vy'(t) &= \vf(t, \vy(t)) \\
            \vy(t_0) &= \vy_0 \\
        \end{aligned}.
        \right.
    \end{equation*}
    Normalmente asumimos que $f$ es continua y localmente Lipschitziana.

    \item[$\vy : I \subset \R \to \R^n$] es la incógnita
    o solución particular de un PVI general de la forma
    \begin{equation*}
        \left\{
        \begin{aligned}
            \vy'(t) &= \vf(t, \vy(t)) \\
            \vy(t_0) &= \vy_0 \\
        \end{aligned}.
        \right.
    \end{equation*}

    \item[$L$] es la constante de lipschitz de la función $\vf$.

    \item[$\vw_i$] es el $i$-ésimo punto generado por
    un método de integración ideal.
    Cuando se trabaja con varios métodos simultáneamente
    se diferencian los puntos con superíndices.

    \item[$\tw_i$] es el $i$-ésimo punto generado por
    un método de integración ejecutado en un ordenador de precisión finita.
    Es decir, $\tw_i \approx \vw_i$, salvo por los errores de redondeo.

    \item[$\delta_i$] es el error por aproximación numérica en el $i$-ésimo paso.
    Es decir, $\delta_i = \vw_i - \tw_i$.

    \item[$h$] es, normalmente, el tamaño del paso de un método de paso fijo.
    
    \item[$\ELT_i$] es el error local de truncamiento (\cref{def:elt}).

    \item[$\vz_i$] es una solución al PVI dado por
    $\vf$ con condición inicial $(t_i, \vw_i)$.
\end{description}
